
\chapter{Apresentação}


%  
 
 O Projeto Pedagógico de um curso é o documento que expressa a sua identidade.
 Tem como finalidade precípua apresentar à comunidade acadêmica como o curso
 caracteriza-se e organiza-se, em função de suas escolhas e percursos, para
 contribuir na formação profissional que se propõe a oferecer aos discentes.
 
 Nesse sentido, o presente documento versa sobre o projeto pedagógico do Curso
 Técnico Subsequente em Informática para Internet do Instituto Federal de
 Educação, Ciência e Tecnologia do Ceará, \textit{campus} Tauá, e está
 fundamentado nas bases legais e nos princípios norteadores que regulamentam a
 educação profissional de nível médio.
 
 
 A idealização deste curso foi feita por meio do projeto de expansão do
 \textit{campus} Tauá, pensando em atender e melhorar seu atendimento ao município
 de Tauá e municípios vizinhos. 
 Foi observado que há na região demanda reprimida de alunos que
 não possuíam dimensões profissionalizantes. Assim, movido pelo seu Plano de Desenvolvimento da Instituição, que tem por
 objetivo expandir as possibilidades de oferta de cursos na comunidade visando
 um melhor desenvolvimento tecnológico e interdisciplinar, o \textit{campus} Tauá, que já tinha o parecer favorável à implantação 
 do Curso Técnico em Informática para Internet, na modalidade subsequente, deu início às tratativas da criação do referido curso.
 
 
 Para o processo de alteração do PPC, foi instituída uma comissão, mediante
 Portaria \nord{2275}/GAB-TAU/DG-TAU/TAUA, DE 05 DE ABRIL DE 2024, composta pelos professores
 Anelise Daniela Schinaider, Antônio Sávio Silva Oliveira, Júlio Serafim Margins, Lucas Ferreira Mendes e Saulo Anderson Freitas de
 Oliveira, a pedagoga Prucina de Carvalho Bezerra,  a  bibliotecária-documentalista 
Analice Fraga de Oliveira, a coordenadora do NEABI Margarida Maria Xavier da Silva, a coordenadora do NAPNE Sharlene Pereira Alves e o coordenador do NUGED Carlos Getúlio de Freitas Maia. Os membros atuaram  em diversas reuniões, observando a legislação vigente, construíram o
 presente documento. Foi dada atenção especial aos Programas de Unidades Didáticas
 e as cargas horárias das disciplinas para gerar compatibilidade com o curso Superior de Tecnologia em Análise e Desenvolvimento de Sistemas, de maneira a melhor distribuir as trilhas de conhecimento junto à complexidade delas pelo curso e à verticalização do Ensino para os concludentes.
 
 

 


 
 % A elaboração deste projeto pedagógico teve como primeiro procedimento
 % metodológico a pesquisa documental nas leis, decretos e resoluções acerca da
 % criação e oferta de cursos subsequentes pelas Instituições Federais. Com
 % isso, delimitou-se a base pedagógica e normativa para o curso subsequente a
 % ser ofertado no Campus Tauá.
 Além disso, contou-se com as orientações pertinentes nas normativas
 institucionais no âmbito dos cursos técnicos, tais como, o Regulamento da
 Organização Didática do IFCE (ROD) e  o Plano de Desenvolvimento Institucional
 do IFCE (PDI) e o Projeto Político-Pedagógico Institucional (PPI).
% \begin{itemize} \setlength\itemsep{0em} \item Regulamento da Organização
% Didática no IFCE -- ROD; \item Plano de Desenvolvimento Institucional do IFCE
% -- PDI; \item Projeto Pedagógico Institucional -- PPI.
% \end{itemize}
 

O documento está organizado em dez (10) seções, a saber: Apresentação, Contextualização da Instituição, Justificativa para a Criação do Curso, Fundamentação Legal, Objetivos do Curso, Organização do Curso, Avaliação do Projeto do Curso, Políticas Institucionais Constantes no PDI, Apoio ao Docente, e, por fim, Infraestrutura.


Inicialmente, nas seções Contextualização da Instituição e Justificativa para a Criação do Curso são descritos um breve histórico da Instituição e do \textit{campus} Tauá, a justificativa para criação do curso e os princípios norteadores regionais que guiam a proposta de implantação deste curso. Em seguida, apresenta-se a fundamentação legal, os objetivos e os itens que compõem a organização do curso, tais como: as formas de ingresso, as áreas de atuação e o perfil esperado do futuro profissional. Logo após, é apresentada a  matriz curricular e seu fluxograma, os aspectos referentes à avaliação da aprendizagem, à prática profissional, ao aproveitamento de conhecimentos, à emissão de diploma, ao perfil docente e ao rodízio nas unidades curriculares. Aborda-se ainda, sobre projetos integradores, atividades complementares, metodologias empregadas no ensino e sua integração \`a pesquisa e extensão.

Logo depois, são abordados aspectos  da avaliação do projeto do curso e as metas que serão oportunizadas dentro do Plano de Desenvolvimento Institucional do \textit{campus} Tauá. Continuando, são elencadas ações estratégicas de apoio ao discente através dos setores existentes, apresentando o corpo docente necessário para a execução do curso. Na sequência, a seção Infraestrutura descreve as instalações e espaços disponibilizados pelo \textit{campus} para as diversas atividades inerentes ao dia-a-dia do curso. Por fim, os anexos detalham os Programas de Unidade Didática (PUDs) das disciplinas que formam a matriz curricular do curso e demais anexos referentes à organização da Instituição e do curso.


\chapter{Contextualização da instituição}

\section{Finalidades do Instituto Federal de Educação, Ciência e Tecnologia do Ceará}


O Instituto Federal de Educação, Ciência e Tecnologia do Ceará (IFCE) é uma Instituição Tecnológica que tem como marco referencial de sua história a evolução contínua com crescentes indicadores de qualidade. A sua trajetória corresponde ao processo histórico de desenvolvimento industrial e tecnológico da Região Nordeste e do Brasil.

A história institucional inicia-se no século XX, quando o então Presidente Nilo Peçanha cria, mediante o Decreto n$^\circ$  7.566, de 23 de setembro de 1909, as Escolas de Aprendizes Artífices, com a inspiração orientada pelas escolas vocacionais francesas, destinadas a atender à formação profissional aos pobres e desvalidos da sorte. O incipiente processo de industrialização passa a ganhar maior impulso durante os anos 40, em decorrência do ambiente gerado pela II Guerra Mundial, levando à transformação da Escola de Aprendizes Artífices em Liceu Industrial de Fortaleza, no ano de 1941 e, no ano seguinte, passa a ser chamado de Escola Industrial de Fortaleza, passando a ofertar formação profissional diferenciada das artes e ofícios, mas orientada para atender às profissões básicas do ambiente industrial e ao processo de modernização do País.

O crescente processo de industrialização, mantido por meio da importação de tecnologias orientadas para a substituição de produtos importados, gerou a necessidade de formar mão de obra técnica para operar esses novos sistemas industriais e para atender às necessidades governamentais de investimento em infraestrutura. No ambiente desenvolvimentista da década de 50, a Escola Industrial de Fortaleza, mediante a Lei Federal n$^\circ$ 3.552, de 16 de fevereiro de 1959, ganhou a personalidade jurídica de Autarquia Federal, passando a gozar de autonomia administrativa, patrimonial, financeira, didática e disciplinar, incorporando a missão de formar profissionais técnicos de nível médio.

Em 1965, passa a se chamar Escola Industrial Federal do Ceará e em 1968, recebe então a denominação de Escola Técnica Federal do Ceará, demarcando o início de uma trajetória de consolidação de sua imagem como instituição de educação profissional, com elevada qualidade, passando a ofertar cursos técnicos de nível médio nas áreas de Edificações, Estradas, Eletrotécnica, Mecânica, Química Industrial, Telecomunicações e Turismo.


O contínuo avanço do processo de industrialização, com crescente complexidade tecnológica, orientada para a exportação, originou a demanda de evolução da Rede de Escolas Técnicas Federais, já no final dos anos 70, para a criação de um novo modelo institucional, surgindo então os Centros Federais de Educação Tecnológica – CEFET's.


A partir da Lei 11.892, de 29 de dezembro de 2008, sancionada pelo então
presidente Luiz Inácio Lula da Silva, passou \`a denominação de Instituto
Federal de Educação, Ciência e Tecnologia do Ceará, mediante integração do
Centro Federal de Educação Tecnológica do Ceará e das Escolas Agrotécnicas
Federais de Crato e de Iguatu, tendo hoje 35 unidades (34 \textit{campi} e Reitoria), distribuídas em todas as
regiões do Estado.
Ao longo da história, os Institutos Federais tornaram-se instituições de
educação superior, básica e profissional, pluricurriculares e multicampi,
especializados na oferta de educação profissional e tecnológica nas diferentes
modalidades de ensino, com base na conjugação de conhecimentos técnicos e
tecnológicos com práticas pedagógicas.

% O campus de Tauá, do Instituto Federal de Educação, Ciência e Tecnologia do Ceará (IFCE), foi inaugurado em 20 de novembro de 2009, como um campus avançado do IFCE de Crateús. Situado em Tauá, município polo da região do sertão dos Inhamuns, distante 334 km de Fortaleza, abrange os municípios de Quiterianópolis, Parambu, Arneiroz e Aiuaba, recebendo também alunos de várias outras regiões, por meio do Sistema de Seleção Unificada (SISU) do Ministério da Educação (MEC), e outros processos seletivos que se fizerem necessários conforme a demanda.


\section{Histórico e Estrutura do IFCE \textit{campus} Tauá}

O \textit{campus} Tauá do IFCE foi inaugurado em 20 de novembro de 2009 como um \textit{campus} avançado do IFCE de Crateús. Situado na cidade de Tauá, município-polo da região Sertão dos Inhamuns, distante 334 $\mathrm{km}$ de Fortaleza, abrange os municípios de Arneiroz, Aiuaba, Parambu e Quiterianópolis \cite{ipece17}, e recebe alunos de várias outras regiões, por meio do Sistema de Seleção Unificada (SISU) do Ministério da Educação (MEC), e outros processos seletivos.

Mesmo antes da inauguração, começaram as tratativas para a definição dos primeiros cursos e serviços a serem ofertados pelo \textit{campus} Tauá. Após uma ampla discussão com a sociedade, ficou definido que, inicialmente, haveria a oferta de dois cursos, um de nível técnico em Agronegócio e outro de nível superior em Tecnologia em Telemática (criado pela Resolução 23/2010 do CONSUP/IFCE, em 31 de maio de 2010).

Procedeu-se à organização de um vestibular e um exame de seleção que, após divulgação e realização, possibilitou o ingresso dos primeiros alunos, ocorrendo inicialmente a oferta de 70 vagas, 35 para cada curso. As primeiras turmas iniciaram as atividades em setembro de 2010 e, semestralmente, novos ingressos foram promovidos, sendo que, para o curso de Telemática, o acesso passou a ser realizado através do SISU/MEC.

Com a adesão ao Programa Nacional de Acesso ao Ensino Técnico e Emprego (PRONATEC), em 2012, o \textit{campus} Tauá passou a ofertar, de forma concomitante, aos alunos do ensino médio da região, um Curso Técnico de Informática, curso este que teve uma oferta única com 40 vagas. Ainda em 2012, o \textit{campus} começou a promover eventos de extensão voltados à divulgação da instituição e fortalecimento das atividades acadêmicas, com destaque para o I Encontro de Tecnologia em Telemática (TECTEL), que passa a ser realizado anualmente pelo curso de Telemática, e a I Semana do Agronegócio, o que inclusive possibilitou o aumento de parcerias com organizações públicas e privadas.

Em 2013, o \textit{campus} Tauá deixou de ser avançado, adquirindo assim, autonomia administrativa, patrimonial, financeira, didático-pedagógica e disciplinar. Nos anos seguintes, tiveram continuidade os investimentos estruturais, como reordenamento de salas, quadra esportiva, laboratórios, e com destaque o novo bloco didático, que possibilitaria a ampliação de cursos e que foi inaugurado em 5 de julho de 2016.

O crescimento da infraestrutura é acompanhado pelo aumento de servidores técnicos administrativos em educação, suprindo as áreas: pedagógica, de assistência estudantil e administrativa, bem como pela chegada de novos docentes.

Um marco das ações do \textit{campus} Tauá, em 2016, foi a sua inserção em programa de intercâmbio internacional, em que, anualmente, o  \textit{campus} tem enviado alunos para cursar um semestre no exterior, atividade que se repetiu em 2017, 2018 e 2019. Em 2016, também houve ofertas de projetos e cursos de extensão: projeto de Xadrez, cursos de planilhas eletrônicas, preparatórios para concursos e para o Enem.

O ano de 2017 foi marcado pela implantação do curso técnico integrado em Redes de Computadores, criado pela Resolução 11/2016 do CONSUP/IFCE, de 4 de março de 2016, possibilitando o \textit{campus} atuar também na oferta do ensino médio. Ademais, com essa nova oferta, o \textit{campus} passa a contar com o aumento significativo de docentes, que, inclusive, reforçam as atividades de extensão. O ano de 2017 culminou com a organização do novo semestre com a nova oferta de turmas do superior em Telemática (via SISU), técnico integrado em Redes de computadores (via edital de seleção), o novo curso de Licenciatura em Letras, com habilitação em Língua Portuguesa e Língua Inglesa e o novo curso técnico Integrado de Agropecuária.


Com o apoio dos docentes e técnicos, o \textit{campus} ofertou em 2018 na vertente extensão, as seguintes atividades:
\begin{alineas}
    \setlength\itemsep{0em}
    \item Projeto de Difusão de Tecnologias de Manejo de Ordenha e Produção e Conservação de Volumosos;
    \item  Projeto Protagonismo Juvenil para a Saúde;
    \item  Projeto Conhecer para Incluir, Capacitação para Educação Inclusiva;
    \item  Projetos de Formação Esportiva (basquete, vôlei e futsal);
    \item  Curso Preparatório para o ENEM;
    \item  Curso Preparatório para os Cursos Técnicos (Pré-Técnico);
    \item  Cursos de Línguas Estrangeiras (Inglês Básico e Espanhol Básico);
    \item  Cursos de Formação Musical (iniciação ao violão e aperfeiçoamento musical).
\end{alineas}

O \textit{campus} Tauá, em 2018, promoveu a I Jornada de Humanidades. Evento este que debateu  gênero e questões raciais. Em seguida, foram realizadas eleições para a para direção-geral, culminando no início do mandato do terceiro diretor da história do \textit{campus}. Ainda em 2018, em fevereiro, foi realizada audiência pública  para definição de cursos a serem ofertados em Tauá. Nutrição, Manutenção Automotiva e \textbf{Informática para Internet}, por exemplo, foram cursos  apontados e votados pelos membros da consulta. 

Além disso,  o ano de 2018 também contou com a participação de mais um aluno enviado à Portugal pelo programa  IFCE Internacional. Por fim, o ano de 2018 culminou com a aprovação da primeira aluna do curso de Telemática, na seleção para mestrado do Programa de Pós-graduação em Sistemas e Computação (PPgSC) da Universidade Federal do Rio Grande do Norte (UFRN).


Em 2019, o IFCE \textit{campus} Tauá venceu etapa estadual de Prêmio de Educação do SEBRAE. O Projeto premiado foi parceria entre os campi de Tauá e Boa Viagem. No segundo semestre daquele ano, o Encontro Pedagógico debate Base Nacional Comum Curricular. Após o início do segundo semestre, diversas ações planejadas no início do ano são executadas no \textit{campus} Tauá:
\begin{alineas}
    \item Participação dos alunos da Feira Agropecuária dos Inhamuns (Inhamunsagro), com apresentações de produtos derivados do leite de cabra;
    \item Maratona Universitário Empreendedor (Sebrae);
    \item  Corredores Digitais (Sebrae);
    \item VII TECTEL, cujo tema principal é a interdisciplinaridade entre tecnologia e agropecuária;
    \item II Jornada de Humanidades;
    \item I Semana de Letras;
    \item II Concurso de Educação Integradora do IFCE, promovido pelos \textit{campi} de Tauá, Boa Viagem e Crateús;
    \item Corrida de Rua Comemorativa do Aniversário de uma Década do \textit{campus} Tauá.
\end{alineas}

No final do ano de 2019, mais especificamente no dia 20 de novembro, foi
comemorada a chegada, há dez anos, do  \textit{campus} Tauá no município.  Para
celebrar uma década de atividades juntamente com todos que fizeram e fazem parte
dessa história, o \textit{campus} preparou uma programação especial. O ano de
2019  encerrou-se com a formatura da primeira turma do curso técnico integrado
em Redes de Computadores e com o  IV Encontro dos Profetas da Chuva dos
Inhamuns.


Em 2020, a pandemia causou
uma pausa abrupta  na programação de  um conjunto de ações voltadas para a comunidade externa.
A transição para o ensino remoto (mas também outros eventos atividades) obrigou a todos passarem por um processo
de digitalização. Até nos acostumarmos com o \textit{novo normal}, muita reflexão sobre a condição da comunidade 
acadêmica foi feita. Inicialmente, o IFCE lançou uma plataforma para a oferta de cursos de formação
inicial e continuada, com o objetivo de capacitar, aperfeiçoar e atualizar
pessoas em todos os níveis de escolaridade, nas mais diversas áreas do
conhecimento. O \textit{campus} Tauá, lançou alguns cursos FICs e  inscrições para a seleção
de bolsistas e voluntários do Programa Institucional de Bolsa de Iniciação
Científica (PIBIC). Também, em 2020, após participar do programa IFCE
Internacional e se formar no curso superior de Tecnologia em Telemática, um dos nossos egressos se preparava parao mestrado em Portugal.
Além disso, eventos com II Festival Cine AVxado, palestras virtuais aconteceram de forma diluída pelo ano.


Em 2021, o campus continuou com iniciativas de Enfrentamento à Covid-19 através
da realização de ações como a doação de cestas básicas e materiais de higiene a
pessoas em situação de vulnerabilidade social, a divulgação de materiais
informativos e educativos nos meios de comunicação do campus, a produção e
distribuição de protetores faciais a instituições de ensino do município e a
elaboração de curso de capacitação sobre medidas de proteção para servidores e
colaboradores. Neste ano, tivemos a aprovação do Curso Técnico em Informática
para Internet com turma inicial prevista para 2022.

De volta ao presencial, em 2022, iniciamos o ano com a solenidade de conclusão
dos cursos técnicos integrados em Agropecuária (a primeira turma) e Redes de
Computadores. Ocorreu, também, o lançamento do Plano de Ação Territorial da
Ovinocaprinocultura de Corte do Banco do Nordeste, junto da   equipe do campus
responsável pelo projeto de Indicação Geográfica da Manta de Carneiro dos
Inhamuns (projeto este com bastante trabalho durante a pandemia).
Adicionalmente, outros eventos, aconteceram durante o ano, a saber, o I Fórum da
Rota do Cordeiro dos Inhamuns, o Programa de Germinação de Ideias e Negócios
Inovadores (PGINI) e a a II Semana do Orgulho LGBTQIA+.

Mais direcionado ao seguimento discente, tivemos a uma aluna do curso superior
de Licenciatura em Letras  selecionada pelo programa IFCE Internacional para
estudar durante seis meses na Universidade de Évora. O intercâmbio deveria ter
acontecido em 2020, mas foi adiado devido à pandemia de Covid 19. Além disso, 05 alunos 
do curso superior de Tecnologia em
Telemática viraram estagiários (de forma remota) da empresa de tecnologia Compass Uol,  no time de \textit{cognitive computing} e desenvolverão atividades
relacionadas a \textit{machine learning}, inteligência artificial e serviços da Amazon.



O campus, em 2023 continuou palco de eventos que fortaleceram as potencialidades da
região dos Inhamuns, como o II Fórum do Empreendedor Apícola dos Sertões dos
Crateús e Inhamuns. Mais ao final de 2023, duas alunas do curso de Licenciatura em Letras do
\textit{campus}   estiveram entre as estudantes que venceram, nas categorias Comunicação e Tecnologia e Produção, o I Prêmio de
Extensão Anna Érika Ferreira Lima Meireles, concurso promovido pela
Pró-Reitoria de Extensão do IFCE que destacou os melhores
extensionistas da instituição.


Mais recentemente, no início de 2024, o \textit{campus} deu mais um passo no acesso à
educação técnica e tecnológica ao receber as primeiras turmas do Curso Técnico
Integrado em Agroindústria, na modalidade Educação de Jovens e Adultos (EJA) e do curso de Tecnologia em Análise e Desenvolvimento de Sistemas. 
Além disso, ocorreu a primeira defesa de trabalho de conclusão de curso do curso de Licenciatura em Letras Português e Inglês.


Como se pode perceber, o \textit{campus} Tauá, com a diversidade formativa que
nele começa a se fortalecer, coloca-se como exemplo viável ao potencial que hoje
possui o IFCE na direção de uma formação autônoma e contextualizada para a
juventude, em face aos desafios postos pelo moderno e competitivo mercado de
trabalho. Logo, este é um terreno no qual todos (professores, técnicos, gestores
e comunitários) podem e devem dar a sua contribuição.



\chapter{Justificativa para a oferta do curso}

Com a popularização da Internet e diversificação de seus serviços, uma demanda
crescente e variada por serviços informatizados está ocorrendo na sociedade como
um todo.  A Internet atualmente faz parte da rotina diária das pessoas. Cada vez
mais tem-se a necessidade de se realizar atividades corriqueiras sem sair de
casa, como por exemplo, fazer compras, realizar pagamento de contas, estudar,
ler um livro, realizar reuniões, dentre outras. Com isso, soluções
computacionais que utilizam internet tornam-se vitais. A pandemia da Covid-19 forçou o mundo todo a passar
por um processo de digitalização bruto e com isso demanda de profissionais de TIC é crescente. 

De acordo com levantamento
da Brasscom (Associação Brasileira das Empresas de Tecnologia da Informação)
realizado em 2019, até o ano de 2024 seria necessário um total de 420 mil
profissionais dessa área. Em nova atualização no ano de 2022, essa projeção foi
praticamente dobrada, indicando a necessidade de 800 mil profissionais de TIC nos
próximos anos devido a aceleração das contratações e a alta demanda por softwares
(CONVERGÊNCIA DIGITAL, 2022). Para atender essa demanda, é necessária a
formação de mais de 100 mil profissionais ao ano. Hoje essa formação está em torno
de 53 mil pessoas, déficit este que já é percebido no mercado de trabalho há alguns
anos (TERRA, 2023).


Neste contexto, novas ocupações estão sendo criadas e outras estão sendo
elevadas em nível de importância. É também neste contexto que atua o
profissional de Informática para Internet, criando sistemas \emph{web} para a
resolução de problemas ou realização de tarefas, acessíveis de diferentes aparelhos.

No Ceará especificamente, segundo a Associação das Empresas Brasileiras de
Tecnologia da Informação, Software e Internet (ASSESPRO), enquanto o desemprego
atinge profissionais em vários segmentos da economia no país, o setor de
Tecnologia da Informação (TIC)  tem muitas vagas ainda não preenchidas para
profissionais qualificados na área \cite{estadao2019}. Segundo a Agência de
Desenvolvimento do Estado do Ceará (ADECE), através da plataforma de
planejamento estratégico Ceará 2050, o setor de TIC foi identificado como uma
das megatendências que afetarão os serviços no Ceará nos próximos anos. Esta é
uma área que está associada à criação de oportunidades em vários setores
econômicos dinâmicos ou de suporte às empresas e que pode potencializar
significativos ganhos de produtividade para o mercado cearense~\cite{adece2019}.

Em acompanhamento a essa tendência, o governo do estado do Ceará está ampliando
o Cinturão Digital \cite{diario2018cinturao}, que vem viabilizando o
funcionamento de diversos projetos e transformando sensivelmente a vida de
milhões de cearenses, especialmente o interior. O Cinturão Digital dota o estado
de um avançadíssimo serviço de transmissão de dados que tem como resultado
prático a melhoria na qualidade e eficiência nos serviços prestados ao cidadão.


Sendo Tauá um dos pontos principais do \emph{backbone} do cinturão digital,
surgem diversas oportunidades de exploração e aproveitamento dos recursos por
ele oferecidos, favorecendo o desenvolvimento sustentável da cidade e abrindo as
portas para que esse município cearense possa se inserir no mercado da TIC de
forma eficiente e competitiva, criando meios de proporcionar o desenvolvimento e
o fortalecimento   de todos os setores, como o agronegócio e o comércio local,
por exemplo. Além de proporcionar abertura para a exploração de novas áreas,
surgindo, com isso,  a necessidade crescente de profissionais qualificados em
informática para a atuação de forma direta e indireta nas tecnologias
proporcionadas pelo Cinturão Digital.



Em virtude da contextualização e das características do IFCE  \textit{campus}
Tauá, que busca um novo parâmetro de desenvolvimento regional para a melhoria da
qualidade de vida, o Curso Técnico Subsequente em Informática para Internet 
configura-se como uma excelente oportunidade, tendo em vista que se caracteriza
por despertar a vocação empreendedora na área de informática, bem como motivar a
participação efetiva na evolução econômica, social e cultural da comunidade.

	
Outro aspecto que norteou a proposição deste Curso foi o aumento do contingente
escolar no ensino médio. As estatísticas revelam uma tendência de forte
aceleração da demanda reprimida de candidatos à matrícula em cursos técnicos em
toda a região de sua abrangência. 

O estudo de potencialidades, apresentado na audiência, considerou questionários
realizados com a população local e informações levantadas em parceria com
secretarias de municípios que compõem a região e diversos órgãos, como o
Instituto Brasileiro de Geografia e Estatística (IBGE), o Sistema Nacional de
Emprego (SINE), o Serviço Brasileiro de Apoio às Micro e Pequenas Empresas
(SEBRAE) e a Câmara de Dirigentes Lojistas (CDL) de Tauá.

Após amplo debate com a sociedade da região dos Inhamuns, a audiência pública,
realizada no dia 07 de março de 2018, concretizou o processo democrático de
escolha e implantação de novos cursos no \textit{campus} de Tauá. O objetivo foi
possibilitar que a comunidade apontasse as qualificações que mais se adequam às
necessidades da região. O evento teve a participação de servidores e alunos do
\textit{campus}, assim como de representantes de secretarias municipais, instituições de
ensino, comércio e diversos outros setores da região. Foi priorizada a oferta de
cursos técnicos subsequentes, voltados a quem já tenha concluído o ensino médio.


Foram definidos onze (11) cursos técnicos (Agropecuária, Alimentos,
Agroindústria, Apicultura, Comércio, Manutenção Automotiva, Eletrotécnica,
Geoprocessamento, \textbf{Informática para Internet}, Nutrição e Dietética e 
Meio ambiente); cinco (5) cursos superiores (Agronomia, Nutrição, Sistemas de
Informação, Gestão Comercial e Alimentos) e duas (2) licenciaturas (Matemática e
História).



Na realidade específica do município de Tauá e microrregião atendida pelo
\textit{campus} Tauá, há diversas escolas municipais que ofertam ensino médio e
que apresentam expressivos números de alunos matriculados, conforme descrito na
\autoref{tab:inep_ibge_matriculas}.


\begin{table}[htpb]
\IBGEtab{
    \caption{Censo Escolar dos municípios limítrofes da Cidade de Tauá: número de matrículas escolares no ensino médio.}
        \label{tab:inep_ibge_matriculas}
}{


	
	    \renewcommand{\arraystretch}{1.3}
\normalsize
	% \begin{tabular}{l r r r r}
	\begin{tabularx}{.97\linewidth}{X}
	\centering
		\begin{tabular}{l r r r r r }
            \toprule
            \cellcolor{gray!20} \textbf{MUNICÍPIO} & \cellcolor{gray!20}\textbf{2019} &  \cellcolor{gray!20}\textbf{2020} & \cellcolor{gray!20}\textbf{2021} & \cellcolor{gray!20}\textbf{2022}  & \cellcolor{gray!20}\textbf{2023} \\

            Aiuaba & 565 &  548 & 575 & 550 & 510 \\
            Arneiroz & 319 & 340 & 344 & 308 & 303\\
            Parambu & 1285 & 1323 & 1432 & 1339 & 1242 \\
            Quiterianópolis & 669 & 705 & 780 & 804 & 881 \\
            Tauá & 2464 & 2375 & 2355 & 2266 & 2334 \\



            \cellcolor{gray!10}\textbf{TOTAL} & \cellcolor{gray!10}\bfseries 5302 & \cellcolor{gray!10}\bfseries 5291 & \cellcolor{gray!10}\bfseries 5486 & \cellcolor{gray!10}\bfseries 5267 & \cellcolor{gray!10}\bfseries 5270 \\
	\bottomrule
	\end{tabular}
        \end{tabularx}
        \vspace{1em}
}
{
    \fonte{Estatísticas Censo Escolar | Inep | 2019 - 2023.}
}
\end{table}


 No entanto, nem todas as escolas da região oferecem educação profissionalizante
 como parte da formação destes discentes. De acordo com a Coordenadoria Regional
 de Desenvolvimento da Educação (CREDE), somente duas\footnote{Endereço das
 Escolas. Disponível em:
 \url{https://educacaoprofissional.seduc.ce.gov.br/endereco-das-escolas/}.
 Acessado em: 17 de março de 2024.} escolas estaduais, a saber, a EEEP Monsenhor
 Odorico de Andrade, em Tauá, e a EEEP Joaquim Filomeno Noronha, em Parambu, e o
 IFCE \textit{campus} Tauá possuem ofertas de cursos técnicos em toda a região
 de abrangência. Nos último editais de seleção de alunos para o ano letivo de
 2024, a EEEP Monsenhor Odorico de Andrade, em Tauá, ofertou 180
 vagas\footnote{EDITAL No. 02/2023 – EEEP MONSENHOR ODORICO DE ANDRADE.
 Disponível em:
 \url{https://www.crede15.seduc.ce.gov.br/wp-content/uploads/sites/133/2023/11/Edital_Matricula__2024_EEEP-MONSENHOR-ODORICO-DE-ANDRADE.pdf}.},
 a EEEP Joaquim Filomeno Noronha, em Parambu, ofertou 180 vagas\footnote{EDITAL
 No. 02/2023 – EEEP JOAQUIM FILOMENO NORONHA. Disponível em:
 \url{https://www.crede15.seduc.ce.gov.br/wp-content/uploads/sites/133/2023/11/Edital_Matricula_2024_EEEP-JOAQUIM-FILOMENO-NORONHA.pdf}.
 Acessado em: 17 de março de 2024.} e o IFCE \textit{campus} Tauá ofertou 100
 vagas\footnote{IFCE - PROCESSO SELETIVO 2024.1 – CURSOS TÉCNICOS – MULTICAMPI
 1. Disponível em:
 \url{https://www.funetec.com/noticias/ver/546}. Acessado em: 29 de jan de
 2021.}. Assim, as estatísticas revelam uma tendência de falta de cobertura
 reprimida de candidatos à matrícula em cursos de técnicos em toda a região, uma
 vez que anualmente somente há 490 vagas para cursos técnicos, representando
 algo em torno de 9\% das matrículas totais.
 
 Essa realidade de estreitamento dos conteúdos educacionais nos currículos do
 ensino médio, o que restringe os discentes em dimensões prático-utilitárias,
 vai em na direção oposta \`a viabilidade de os mesmos conquistarem seu espaço
 no mercado de trabalho e progredirem com sucesso.
 
 Como alternativa a esse cenário de discentes com horizontes profissionais
 limitados, a busca por um equilíbrio nos percursos educacionais  faz-se
 necessária. Portanto, a oferta do Curso Técnico Subsequente em Informática para
 Internet objetiva oportunizar ao discente uma formação sólida e atualizada,
 aliada ao desenvolvimento de competências que possibilitarão o atendimento de
 várias demandas profissionais.
 
A escolha da modalidade subsequente também vai ao encontro desse sentimento de
disponibilidade de oportunidades. Por ser menos restritiva que a versão
concomitante e integrada, a modalidade subsequente viabiliza para quem já
concluiu o ensino médio o acesso à Educação Profissionalizante. Portanto, esses
alunos são potenciais candidatos ao Curso Técnico Subsequente em Informática
para Internet.


\section{A proposta do Curso Técnico Subsequente em Informática para Internet}

Mediante os sinais observados no mercado atual, a carência de mão de obra especializada, a baixa cobertura de dimensões profissionalizantes na formação regional dos discentes e visando atender às necessidades que surgem nessa nova configuração econômica de Tauá, o presente projeto visa \`a implantação do Curso Técnico Subsequente em Informática para Internet no IFCE  
\textit{campus} Tauá objetivando formar profissionais para atender às demandas da área na região do Inhamuns.

% Nota-se que a cidade de Tauá e a região dos Inhamuns suportam e carecem de
% cursos técnicos com ênfase na área de tecnologia, e neste caso,
A proposta do Curso Técnico Subsequente em Informática para Internet vem como
uma oportunidade de formação complementar aos alunos que irão se formar nos
próximos anos no ensino médio e aos formados que não obtiveram a dimensão
profissionalizante durante a formação. Assim, fornecendo condições a esses
alunos de também conquistarem seu espaço no mercado de trabalho e desenvolvam-se
com sucesso.


O IFCE \textit{campus} Tauá, ciente da importância do seu papel diante do
cenário de transformações que hoje se apresenta no mundo do trabalho, está se
preparando para enfrentar tal tarefa com qualidade, reformulando seus
currículos, reinterpretando o seu relacionamento com o segmento produtivo e
buscando atender às novas demandas de oportunidades formativas que vão surgindo
na região. Dessa forma, a criação do curso subsequente supracitado visa suprir
essa carência do mercado, além de elevar o potencial competitivo do IFCE,
tornando-o referência no segmento.




