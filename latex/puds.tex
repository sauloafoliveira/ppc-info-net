
\newcommand{\cabecalho}{
\begin{center}
{	\vspace{-5em}
	\includegraphics[width=0.36\textwidth]{logo_taua_pb.png}

 }
        \ABNTEXchapterfont
        \footnotesize
       
       % \includegraphics[width=.15\linewidth]{ifce_peb.png}\\
        DEPARTAMENTO DE ENSINO\\
        COORDENAÇÃO DO CURSO TÉCNICO  EM INFORMÁTICA PARA INTERNET\\
        PROGRAMA DE UNIDADE DIDÁTICA -- PUD
\end{center}
}


\newcommand{\naopresencial}{

Quanto às atividades não presenciais, as mesmas serão orientadas e
acompanhadas pelo(a) docente da disciplina. Nelas, a avaliação deve permitir ao docente
compreender como o aluno elabora e constrói seu próprio conhecimento. Neste
caso, o acompanhamento do processo de ensino e aprendizagem discente se dá através de
ambientes virtuais de aprendizagem e/ou outros sistemas computacionais
apropriados que possam facilitar o acompanhamento, verificação e validação das
atividades.

Observa-se que as aulas criadas para fins de realização de atividades não
presenciais não devem ser consideradas para controle de frequência do discente.
São registradas as faltas dos estudantes somente quando se ausentarem das aulas
presenciais.
}

\setlist[itemize]{noitemsep, topsep=0pt}

\small

\renewcommand{\arraystretch}{1.3}

\newcommand{\pudinfo}[8]{
	\newpage
	%\section{#2}
    \noindent\begin{tabularx}{\linewidth}{ | X X X X X X | } 

    \multicolumn{6}{p{\textwidth-2\tabcolsep}}{\cabecalho} \\
    
    
	\multicolumn{6}{p{\textwidth-2\tabcolsep}}{\cellcolor{gray!10}DISCIPLINA: \bfseries #2} \\
	
	\multicolumn{2}{p{.33\textwidth-2\tabcolsep}}{C\'odigo: \bfseries #1 }
	\multicolumn{2}{p{.33\textwidth-2\tabcolsep}}{Carga horária total: \bfseries \MULTIPLY{20}{#6}{\solution} \solution{h}}  &
	\multicolumn{2}{p{.34\textwidth-2\tabcolsep}}{Créditos: \bfseries #6} \\
	
	\multicolumn{2}{p{.33\textwidth-2\tabcolsep}}{Nível: \bfseries Técnico }
	\multicolumn{2}{p{.33\textwidth-2\tabcolsep}}{Semestre: \bfseries #7}  &
	\multicolumn{2}{p{.33\textwidth-2\tabcolsep}}{Pré-requisitos: \bfseries #3} \\[0.25em]

	\ifthenelse{
		
		\multicolumn{2}{p{.33\textwidth-2\tabcolsep}}{} &
		\multicolumn{2}{p{.33\textwidth-2\tabcolsep}}{Teórica: \bfseries #4{h} } & 
		\multicolumn{2}{p{.33\textwidth-2\tabcolsep}}{Prática: \bfseries #5{h} } \\
		
		\multicolumn{2}{p{.33\textwidth-2\tabcolsep}}{} &
		\multicolumn{2}{p{.33\textwidth-2\tabcolsep}}{Presencial: \bfseries \ifthenelse{\equal{2}{#6}}{32}{64}h } & 
		\multicolumn{2}{p{.33\textwidth-2\tabcolsep}}{Distância: \bfseries 0h } \\
		
		\multicolumn{2}{p{.33\textwidth-2\tabcolsep}}{Carga Horária} &
		\multicolumn{2}{p{.33\textwidth-2\tabcolsep}}{Profissional: \bfseries \ifthenelse{\equal{TSII.203}{#1} \OR \equal{TSII.204}{#1} \OR \equal{TSII.304}{#1}}{12}{0}h } & \\
		
		\multicolumn{2}{p{.33\textwidth-2\tabcolsep}}{} &
		\multicolumn{4}{p{.67\textwidth-2\tabcolsep}}{Atividades não presenciais: \bfseries \ifthenelse{\equal{2}{#6}}{8}{16}h }  \\
	}
	\multicolumn{2}{p{.33\textwidth-2\tabcolsep}}{} &
	\multicolumn{4}{p{.67\textwidth-2\tabcolsep}}{Extensão: \bfseries 0{h} }  \\
	 
	
% 	
%     \multicolumn{1}{p{.16\textwidth-2\tabcolsep}}{\cellcolor{gray!40}\bfseries\large #1} & \multicolumn{5}{p{.84\textwidth-2\tabcolsep}}{\cellcolor{gray!30}\bfseries\uppercase{#2}} \\ 
%     
%     \multicolumn{2}{ p{.33\textwidth-2\tabcolsep} }{\cellcolor{gray!10}\bfseries Carga Horária}  &
%     \multicolumn{2}{ p{.33\textwidth-2\tabcolsep} }{\cellcolor{gray!10}\bfseries CH Teórica} & 
%     \multicolumn{2}{ p{.34\textwidth-2\tabcolsep} }{\cellcolor{gray!10}\bfseries CH Prática}   \\
%     
% 
%     
%     \multicolumn{2}{ p{.33\textwidth-2\tabcolsep} }{ \MULTIPLY{20}{#6}{\solution} \solution } &
%     \multicolumn{2}{ p{.33\textwidth-2\tabcolsep} }{#4} & 
%     \multicolumn{2}{ p{.34\textwidth-2\tabcolsep} }{#5} \\
%     
%     \multicolumn{2}{ p{.33\textwidth-2\tabcolsep} }{\cellcolor{gray!10}\bfseries Número de Créditos}  &
%     \multicolumn{2}{ p{.33\textwidth-2\tabcolsep} }{\cellcolor{gray!10}\bfseries Código Pré-Requisito} & 
%     \multicolumn{2}{ p{.34\textwidth-2\tabcolsep} }{\cellcolor{gray!10}\bfseries Semestre}   \\
%     
%     \multicolumn{2}{ p{.33\textwidth-2\tabcolsep} }{#6} &
%     \multicolumn{2}{ p{.33\textwidth-2\tabcolsep} }{\small #3} & 
%     \multicolumn{2}{ p{.34\textwidth-2\tabcolsep} }{#7} \\ 
%     
    
	\end{tabularx}



	

}

\newenvironment{pud}{
	\normalsize
	\SingleSpacing
	\setlength{\parindent}{0em}
	
}{
	\vspace{0.5em}
	\noindent\begin{tabularx}{\linewidth}{ | X X | }
		
	
	\multicolumn{2}{p{\textwidth-2\tabcolsep}}{
		\cellcolor{gray!20} 
		\begin{tabular}{p{.45\linewidth} p{.45\linewidth}}
			\multicolumn{1}{c }{\newline\,Coordenador do Curso} & \multicolumn{1}{c}{\newline\,Coordenadoria Técnico--Pedagógica}  \\
			\\
			\\
			\hrulefill & \hrulefill
		\end{tabular}
    }
    \end{tabularx}
}

\newenvironment{ementa}{

	\noindent\begin{tabularx}{\linewidth}{ X }
		\cellcolor{gray!10}\textbf{EMENTA}
	\end{tabularx}\\[5pt]	
	\noindent}{\newline}

\newenvironment{objetivos}{
	\vspace{0.5em}
	\noindent\begin{tabularx}{\linewidth}{ X }
		\cellcolor{gray!10}\textbf{OBJETIVOS}
	\end{tabularx}\\[5pt]	
	\noindent}{\newline}

\newenvironment{programa}{
	\vspace{0.5em}
	\noindent\begin{tabularx}{\linewidth}{ X }
		\cellcolor{gray!10}\textbf{PROGRAMA}
	\end{tabularx}	
	\noindent}{\newline}

\newenvironment{metodologia}{
	\vspace{0.5em}
	\noindent\begin{tabularx}{\linewidth}{ X }
		\cellcolor{gray!10}\textbf{METODOLOGIA}
	\end{tabularx}\\[5pt]	
	\noindent}{\newline}

\newenvironment{avaliacao}{
	\vspace{0.5em}
	\noindent\begin{tabularx}{\linewidth}{ X }
		\cellcolor{gray!10}\textbf{AVALIAÇÃO}
	\end{tabularx}\\[5pt]	
	\noindent}{\newline}

\newenvironment{recursos}{
	\vspace{0.5em}
	\noindent\begin{tabularx}{\linewidth}{ X }
		\cellcolor{gray!10}\textbf{RECURSOS}
	\end{tabularx}\\[5pt]	
	\noindent}{\newline}

\newenvironment{bibbasica}{
	\vspace{0.5em}
	\noindent\begin{tabularx}{\linewidth}{ X }
		\cellcolor{gray!10}\textbf{BIBLIOGRAFIA BÁSICA}
	\end{tabularx}
	\vspace{-1em}
	\begin{flushleft}	
	\begin{itemize}[label={},leftmargin=0.1em, itemsep=0.75em]}{
	\end{itemize}
	\end{flushleft}
}

\newenvironment{bibcomplementar}{
	\vspace{0.5em}
	\noindent\begin{tabularx}{\linewidth}{ X }
		\cellcolor{gray!10}\normalsize\textbf{BIBLIOGRAFIA COMPLEMENTAR}
	\end{tabularx}
	\vspace{-1em}
	\begin{flushleft}	
	\begin{itemize}[label={},leftmargin=0.1em, itemsep=0.75em]}{
	\end{itemize}
	\end{flushleft}
}

\begin{pud}
	
	\pudinfo{TSII.101}{Introdução à Computação}{---}{20}{20}{2}{1$^\circ$ Semestre}
	
	\ementa
	Visão geral do Curso de Informática para Internet. Princípios fundamentais da Computação. Noções de arquitetura de computadores. Funcionamento das linguagens de programação.

	\objetivos
	Conhecer os componentes de hardware que formam os dispositivos computacionais e identificar o que estes componentes afetam no desempenho do software.
	\begin{itemize}
	  \item Distinguir as áreas de atuação e os recursos utilizados pelos profissionais da área de análise e desenvolvimento de sistemas;
	  \item Conhecer o funcionamento básico dos subsistemas que integram o computador;
	  \item Reconhecer e descrever sistemas digitais e componentes fundamentais;
	  \item Discorrer sobre as principais abordagens para a representação de algoritmos e tradução de códigos-fontes nos dispositivos computacionais;
	  \item Identificar novos temas relacionados a tecnologias emergentes relacionadas à computação.
	\end{itemize}
	
	
	
	\programa	
	\begin{description}[itemsep=0em]
	   
	   \item[UNIDADE I:]  Visão geral do Curso de Informática para Internet;
	   \begin{enumerate}[itemsep=0em, topsep=0em]
	     \item Histórico do curso;
	     \item Características e diferenças dos cursos da área de computação;
	     \item Objetivos gerais do curso, competências, habilidades e perfil do egresso;
	     \item Organização curricular do curso no IFCE \textit{campus} Tauá.
	   \end{enumerate}
	   
	   \item[UNIDADE II:] Fundamentos da Computação;
	   \begin{enumerate}[itemsep=0em, topsep=0em]
	     \item História da computação;
	     \item Hardware e Software.
	   \end{enumerate}
	   
	    \item[UNIDADE III:]  Noções de Arquitetura de Computadores;
	   \begin{enumerate}[itemsep=0em, topsep=0em]
	     \item Organização de computadores;
	     \item Representação de dados;
	     \item Operações matemáticas sobre números binários e hexadecimais;
	     \item Representação de dados em sistemas computacionais.
	   \end{enumerate}
	   
	   	    
	    \item[UNIDADE IV:] Funcionamento das Linguagens de Programação.
	    \begin{enumerate}[itemsep=0em, topsep=0em]
	     \item Lógica computacional;
	     \item Linguagens de Programação;
	     \item Interpretação e compilação de programas.
	   \end{enumerate}
	   
	\end{description}
	
	
	\metodologia            	
       Aulas expositivas e interativas com uso de recursos audiovisuais;
       Atividades em grupo e prática de codificação de algoritmos em linguagem computacional.
       Atividades práticas no laboratório de codificação de programas.
	
	\recursos
	Data-show, pincel e quadro branco, aparelho de som, laboratório de informática e
dicionários.
	
\avaliacao A avaliação é realizada de forma processual e cumulativa utilizando
os instrumentos de avaliação especificados pelo Regulamento de Organização
Didática em seu art. 94 \S~1$^\circ$, conforme for mais adequado. A frequência é
obrigatória, respeitando os limites de ausência previstos em lei.
\naopresencial
	
	\begin{bibbasica}
		\item CARVALHO, André C. P. L. F. de, LORENA, Ana Carolina. \textbf{Introdução à
computação}: hardware, software e dados. Rio de Janeiro: Editora LTC, 2017. 182
p. ISBN 9788521631071.
		\item FOROUZAN, B; MOSHARRAF, F. \textbf{Fundamentos da Ciência da Computação}. 2
ed. São Paulo: Cengage Learning. 2011.560 p. ISBN 9788522110537.
		\item TANENBAUM, A. S. \textbf{Organização estruturada de computadores}. 6.ed. São
Paulo: Pearson Prentice Hall, 2013. 605 p. ISBN 9788581435398.
	 	 
	\end{bibbasica}
	
	\begin{bibcomplementar}
	
		\item SILBERSCHATZ, A. et. al. \textbf{Fundamentos de sistemas operacionais}. 9.ed. Rio de Janeiro: LTC, 2013. 508 p. ISBN 9788521629399.
		\item SCHILDT, Herbert. \textbf{C}: completo e total. 3. ed. rev. e atual São Paulo: Pearson
		Makron Books, 1997. 827 p. ISBN 9788534605953.
		\item SOARES, Walace; FERNANDES, Gabriel. \textbf{Linux}: fundamentos. São Paulo: Érica,
		2010. 206 p. ISBN 9788536503219.
		\item STALLINGS, William. \textbf{Arquitetura e organização de computadores}. 8. ed. São
		Paulo: Pearson Prentice Hall, 2010. 624 p. ISBN 9788576055648.
		\item TANENBAUM, Andrew S. \textbf{Sistemas operacionais modernos}. 4. ed. São Paulo:
		Prentice-Hall, 2016. 758 p. ISBN 9788543005676.
		
	\end{bibcomplementar}
	
	
\end{pud}


	
	\begin{pud}
	
	\pudinfo{TSII.102}{Lógica de Programação}{---}{40}{40}{4}{1$^\circ$ Semestre}
	
	\ementa
	Linguagens de baixo e alto nível, interpretadores e compiladores, variáveis e tipos de dados, operadores, expressões, estruturas de controle de fluxo, processamento de strings, funções e métodos, vetores e matrizes, arquivos e recursão.
	
	\objetivos	
	Desenvolver a capacidade de criar programas para a solução de problemas, usando os fundamentos da programação estruturada.
	
	\begin{itemize}
		\item  Conhecer os conceitos de algoritmos, linguagens de programação de baixo nível e alto nível, compilação e interpretação.
		Identificar os tipos de dados elementares e os operadores relacionados.
		
		\item Conhecer variáveis,  expressões, precedência de operadores e conversões de tipos.
		
		\item Aprender comandos de entrada e saída de dados.
		
		\item Conhecer as principais estruturas de controle de fluxo de execução: estruturas de decisão tipo if-else, estruturas de repetição tipo for e while, comandos break e continue.
		
		\item Manipular dados armazenados em vetores e matrizes.
		
		\item Elaborar funções e métodos usando conceitos de modularização, passagem de parâmetros, variáveis locais e globais e recursão.
		
		\item Utilizar arquivos para armazenar e recuperar dados.

		\item Criar funções que são definidas em termos de si mesmas usando recursão.
	\end{itemize}
	
	
	
	\programa	
	\begin{description}[itemsep=0em]
	   \item[UNIDADE I:] Introdução;
	    \item[UNIDADE II:] Tipos de dados;
	    \item[UNIDADE III:]  Variáveis e expressões;
	    \item[UNIDADE IV:]  Entrada e saída;
	    \item[UNIDADE V:]  Controle de fluxo de execução (condicionais e estruturas de repetição);
	    \item[UNIDADE VI:]  \textit{Strings} (cadeias de caracteres);
	    \item[UNIDADE VII:]  Listas;
	   \item[UNIDADE VIII:] Funções (métodos) e Arquivos.
	\end{description}
		
		% 
	\metodologia            	
       Aulas expositivas e interativas com uso de recursos audiovisuais;
       Atividades em grupo e prática de codificação de algoritmos em linguagem computacional.
       Atividades práticas no laboratório de codificação de programas.
	
	
	\avaliacao	
	A avaliação é realizada de forma processual e cumulativa utilizando os instrumentos de avaliação especificados pelo Regulamento de Organização Didática em seu art. 94 \S~1$^\circ$, conforme for mais adequado. A frequência é obrigatória, respeitando os limites de ausência previstos em lei.
	
	\begin{bibbasica}
		\item ALVES, William Pereira. \textbf{Lógica de programação de computadores}: ensino didático. São Paulo: Érica, 2010. 176 p. Bibliografia. ISBN 9788536502892. 
		\item FORBELLONE, André Luiz Villar. \textbf{Lógica de programação}: a construção de algoritmos e estruturas de dados. 3.ed. São Paulo: Pearson Prentice Hall, 2005.  ISBN 9788576050247. (BVU)
	 	\item XAVIER, Gley Fabiano Cardoso. \textbf{Lógica de programação}. 13. ed. rev. e atual São Paulo: Senac, 2014. 318 p. (Nova série informática). ISBN 9788539604579.
	 	 
	\end{bibbasica}
	
	\begin{bibcomplementar}
	
		\item ASCENCIO, Ana Fernanda Gomes. \textbf{Fundamentos da programação de computadores}: algoritmos, Pascal, C/C++ e Java. 2. ed. São Paulo: Pearson, 2010. 434 p. Inclui bibliografia. ISBN 9788576051480.
 (BVU)
		\item DEITEL, Paul; DEITEL, Harvey. \textbf{Java}: como programar. Tradução de Edson Furmankiewicz. Revisão técnica de Fábio Luis Picelli Lucchini. 8. ed. São Paulo: Pearson Prentice Hall, 2010. 1144 p. ISBN 9788576055631. (BVU)
	    	

    	\item FARREL, Joyce. \textbf{Lógica e design de programação}. Tradução de André Schifnagel Avrichir. Revisão técnica de Robert Joseph Didio. 5. ed. São Paulo: Cengage Learning, 2010. 416 p. Inclui bibliografia. ISBN 9788522107575.
    	\item GUEDES, Sérgio (Org.). \textbf{Lógica de programação algorítmica}. São Paulo: Pearson Education do Brasil, 2014.  ISBN 9788543005546. (BVU)
    	\item PEREIRA, Silvio do Lago. \textbf{Algoritmos e lógica de programação em C}: uma abordagem didática. São Paulo: Érica, 2010. 190 p. Bibliográfia. ISBN 9788536503271.
		
	\end{bibcomplementar}
\end{pud}


\begin{pud}

	\pudinfo{TSII.103}{Inglês Técnico}{---}{30}{20}{2}{1$^\circ$ Semestre}
	
	\ementa 
	Aspectos fundamentais da gramática de língua inglesa. Leitura, análise e interpretação de textos técnicos. Estratégias de leitura em língua estrangeira.

	\objetivos
	Compreender textos escritos em diferentes gêneros textuais em língua inglesa, especialmente aqueles necessários ao desempenho de sua profissão.

	\begin{itemize}
	  \item Desenvolver a competência leitora em língua estrangeira;
	  \item Ler e interpretar textos de sua área de atuação profissional escritos em língua inglesa.
	\end{itemize}
	
	\programa
	\begin{description}[itemsep=0em]
		\item[UNIDADE I:] Leitura para Compreensão Geral.
	    \begin{enumerate}[itemsep=0em, topsep=0em]
	     \item Fundamentos básicos;
	     \item Informação não-verbal;
	     \item Previsão e evidências tipográficas;
	     \item Skimming;
	     \item Seletividade;
	     \item Palavras cognatas e falso-cognatas;
	     \item Uso estratégico do dicionário.
	   \end{enumerate}
	   
	   \item[UNIDADE II:] Leitura para Compreensão das Ideias Principais
	    \begin{enumerate}[itemsep=0em, topsep=0em]
	     \item Scanning;
	     \item Inferência contextual;
	     \item Summarizing (outlining, concept maps, taking notes);
	     \item Estrutura da oração (grupos nominais e verbais);
	     \item Coerência e coesão;
	     \item Marcadores Discursivos.
	   \end{enumerate}
	   
	   \item[UNIDADE III:] Leitura para compreensão de detalhes.
	    \begin{enumerate}[itemsep=0em, topsep=0em]
	     \item Formação de palavras;
	     \item Leitura crítica;
	     \item Sintagma Nominal e Verbal.
	   \end{enumerate}
	   
	   \item[UNIDADE IV:] Tópicos Gramaticais.
	    \begin{enumerate}[itemsep=0em, topsep=0em]
	     \item Simple present e present continuous;
	     \item Simple Past (regular e irregular verbs);
	     \item Present perfect e past perfect;
	     \item Immediate future e simple future;
	     \item Modal Verbs.
	   \end{enumerate}
	\end{description}
	
	\bibbasica
		\item ALMEIDA, Rubens Queiros de. \textbf{As palavras mais comuns da Língua Inglesa}. São Paulo: Novatec, 2003.312 p. ISBN 97885575220373.
		\item BROWN, H. Douglas. \textbf{Teaching by Princinples}: An Interactive Approach to Language Pedagogy. 2ed. New York: Longman, 2001.
		\item HORNBY, A. S. \textbf{Oxford advanced learners Dictionary of Current English}. 7. ed. Oxford (Inglaterra): Oxford University Press, 2007. 1779 p., Il. + Inclui CD-ROM. ISBN 9780194001168 (Broch.).
	
	\begin{bibcomplementar}
		\item DIAS, Reinildes. \textbf{Inglês Instrumental}: leitura crítica: uma abordagem construtiva. 3. ed. revista e ampliada. Belo Horizonte, UFMG, 2002.
		\item GLENDINNING, Eric H.; MCEWAN, John. \textbf{Basic english for computing}. Oxford (Inglaterra): Oxford University Press, 2012. 136 p. ISBN 9780194574709.
		\item LONGMAN. \textbf{Dicionário escolar}: inglês-português, português-inglês. 2. ed. Inglaterra: Pearson, 2009. 770 p. ISBN 9788576592754.
		\item MUNHOZ, Rosângela. \textbf{Inglês instrumental}: estratégias de leitura: módulo I. São Paulo: Textonovo, 2004. 111 p. ISBN 8585734367.
		\item MURPHY, Raymond; SMALZER, William R.; CHAPPLE, Joseph. \textbf{English grammar in use intermediate}: self-study reference and practice for students of North American English: with answers. 4th. ed. Cambridge (England): Cambridge University Press, 2018. 374 p. ISBN 9780521189392.
	
	\end{bibcomplementar}
		
	
\end{pud}



\begin{pud}
	\pudinfo{TSII.104}{Tecnologias WEB}{---}{40}{40}{4}{1$^\circ$ Semestre}
	
\ementa Introdução à Internet e World Wide Web: histórico da internet e web,
conceitos básicos de arquitetura cliente-servidor, serviços da internet;
Hypertext Markup Language (HTML): estrutura, semântica, elementos, boas
práticas, multi pages websites; Cascading Style Sheets (CSS): especificidade,
seletores, elementos, CSS Resets, media queries; Estruturação para a
apresentação da informação: box model, posicionamento com floats, grids; Efeitos
gráficos e animações: animações, transições e transformações CSS; Framework para
front-end: introdução e instalação, CSS responsivo, Grid responsivo; Design da
experiência de usuário (UX): Princípios de IHC, princípios e elementos da UX,
requisitos de usabilidade, acessibilidade na web, padrões de interação e
navegação na Web, layout e composição, avaliação de usabilidade.
	
\objetivos \textbf{OBJETIVO GERAL}:
	
	Construir páginas WEB observando as tecnologias mais atuais e as melhores práticas de construção e formatação de seus elementos, focando na aplicação de práticas e técnicas de Design da Experiência de Usuário na construção de Interfaces Humano-Computador (IHC).
	\newline\\
	\textbf{OBJETIVOS ESPECÍFICOS}:
	\begin{itemize}
		\item Compreender os mecanismos elementares de funcionamento da Internet e da Web; 
		\item Reconhecer a diferença entre linguagens de Marcação, Formatação e Dinâmicas;   
		\item Entender o processo de projeto e produção de front-end para internet;
		\item Adquirir competências sobre marcação com a tecnologia HTML5; 
		\item Adquirir competências sobre marcação com a tecnologia CSS3;
		\item Conhecer ferramentas, técnicas e frameworks para o desenvolvimento de Interfaces Web com tecnologias do lado cliente;
		\item Adquirir competências sobre Design da Experiência do Usuário e IHC para web.
	\end{itemize}
	
	
	\programa
	\begin{description}[itemsep=0em]
		\item[UNIDADE I:] Introdução à Internet e World Wide Web (WWW);
	         \begin{enumerate}[itemsep=0em, topsep=0em]
				\item O Ambiente Web: Histórico da Internet e Web;
				\item Conceitos Básicos da Arquitetura Cliente-Servidor; 
                \item Serviços da Internet.
                
            \end{enumerate}
        \item[UNIDADE II:] Hypertext Markup Language (HTML);
	         \begin{enumerate}[itemsep=0em, topsep=0em]
				\item Estrutura;
                \item Semântica;
                \item Elementos  (\textit{block} e \textit{inline}, de texto, de estrutura, aninhamento,  \textit{links}, listas, tabelas, formulários);
                \item Boas práticas;
				\item \textit{Multi Page Websites}.	
                
            \end{enumerate}
            
         \item[UNIDADE III:] \textit{Cascading Style Sheets};
	         \begin{enumerate}[itemsep=0em, topsep=0em]
				\item Especificidade;
				\item Seletores;
                \item  Cores;
                \item  Comprimentos;
                \item Tipografia;
                \item  Background e Gradientes;
                \item  CSS Resets;
                \item  \textit{Media Queries}.        
            \end{enumerate}
            
         \item[UNIDADE IV:] Estruturação para a apresentação da informação;
	         \begin{enumerate}[itemsep=0em, topsep=0em]
 				\item \textit{Box model};
				\item Posicionamento com \textit{floats};
				\item \textit{Grids}.         	
                
            \end{enumerate}
            
          \item[UNIDADE V:] \textit{Frameworks} para \textit{front-end};
	         \begin{enumerate}[itemsep=0em, topsep=0em]
				\item Introdução e instalação;
				\item CSS responsivo;
				\item \textit{Grid} responsivo.                
            \end{enumerate}
            
          \item[UNIDADE VI:]  Design da experiência de usuário;
	         \begin{enumerate}[itemsep=0em, topsep=0em]
  				\item Princípios de IHC (Ergonomia e Usabilidade, e Engenharia de Usabilidade);
				\item Principios e elementos da UX;
				\item Requisitos de usabilidade;
				\item Acessibilidade na Web;
				\item Padrões de interação e navegação na Web;
				\item Layout e composição;
				\item Avaliação de usabilidade.
	        
                
            \end{enumerate}
	\end{description}
	
	
		
	\metodologia	
    A disciplina contará com aulas expositivas dialogadas, atividades práticas em laboratório e a realização de trabalhos em equipe e individuais, além da análise e discussão de estudos de caso e a aplicação de metodologias ativas para promover a construção do conhecimento no estudante.
    

	\avaliacao
	A avaliação é realizada de forma processual e cumulativa utilizando os instrumentos de avaliação especificados pelo Regulamento de Organização Didática em seu art. 94 \S~1$^\circ$, conforme for mais adequado. A frequência é obrigatória, respeitando os limites de ausência previstos em lei.
	
	\begin{bibbasica}
		
		\item CYBIS, Walter; BETIOL, Adriana Holtz; FAUST, Richard. \textbf{Ergonomia e Usabilidade}: Conhecimentos, Métodos e Aplicações. 2 ed. São Paulo: Novatec, 2010. ISBN 9788575222324.
	    \item FLATSCHART, Fábio. \textbf{HTML5}: Embarque imediato. Rio de Janeiro: Brasport, 2011.  ISBN 9788574525778. (BVU)
	    \item MANZANO, José Augusto N. G.; TOLEDO, Suely Alves de. \textbf{Guia de orientação e desenvolvimento de sites HTML, XHTML, CSS e Javascript/JScript}. 2. ed. São Paulo: Érica, 2010. ISBN 9788536501901.

	\end{bibbasica}
	
	\begin{bibcomplementar}
	
 		\item CHARK, Andrew. \textbf{Como criar sites persuasivos}: clique aqui. São Paulo: Pearson, 2004. ISBN 9788534615112. (BVU)
        \item NIELSEN, Jakob. \textbf{Projetando Websites.} Rio de Janeiro: Campus, 2000. ISBN 85-352-0656-6.
        \item OLIVIERO, Carlos A. J. \textbf{Faça Um Site HTML 4.0}: conceitos e aplicações. São Paulo: Erica, 2007. ISBN 9788536501635.
        \item SEGURADO, Valquiria Santos (Org.). \textbf{Projeto de interface com o usuário}. São Paulo: Pearson Education do Brasil, 2015. Livro. 195 p. ISBN 9788543017303. (BVU)
        \item TERUEL, Evandro Carlos. \textbf{Web total}: desenvolva sites com tecnologias de uso livre: prático e avançado. São Paulo: Érica, 2009. 336p. Bibliografia. ISBN 9788536502328.	

	\end{bibcomplementar}
	
\end{pud}


%%REDES DE COMPUTADORES


\begin{pud}
	\pudinfo{TSII.105}{Redes de  Computadores}{---}{40}{40}{4}{1$^\circ$ Semestre}
	
	\ementa
	Princípios de Comunicação de dados, Topologias, Arquiteturas de redes de computadores, Nível físico, Nível de enlace, Padrões para nível físico e de enlace, Nível de rede, Ligação Inter--Redes, Nível de aplicação e atividade prática em laboratório físico ou virtual.
	
	\objetivos
	Propiciar ao aluno o entendimento do funcionamento básico da comunicação digital de dados; Compreender os modelos de referências, protocolos e serviços básicos de redes de computadores.
	

	\programa
	\begin{description}[itemsep=0em]
		\item[UNIDADE I:] Introdução à redes de computadores; 
	         \begin{enumerate}[itemsep=0em, topsep=0em]
	            \item Histórico;
	            \item Aplicações;
	            \item Tecnologias e Topologias de Rede;
	            \item Arquitetura de Rede: Modelos OSI, TCP/IP e Híbrido.
            \end{enumerate}
            
        \item[UNIDADE II:] Conceitos básicos em redes de computadores;
	         \begin{enumerate}[itemsep=0em, topsep=0em]
                \item Protocolo e encapsulamento;
                \item Atraso;
                \item Erros;
                \item Vazão.
            \end{enumerate}
            
         \item[UNIDADE III:]  Camada de enlace de dados;
	         \begin{enumerate}[itemsep=0em, topsep=0em]
                \item Terminologia e funções;
                \item  Endereçamento MAC;
                \item  Detecção e correção de erros;
                \item  Protocolos de acesso ao meio;
                \item  Protocolo ARP;
                \item  \textit{Switch};
                \item  Protocolo Ethernet.        
            \end{enumerate}
            
         \item[UNIDADE IV:] Camada de  rede;
	         \begin{enumerate}[itemsep=0em, topsep=0em]
                \item Serviços da camada de rede;
                \item Modelos de serviço de Rede;
                \item Protocolo IP;
                \item Endereçamento IP;
                \item Roteamento.
            \end{enumerate}

         \item[UNIDADE V:] Camada de  transporte;
	         \begin{enumerate}[itemsep=0em, topsep=0em]
                \item Funções e serviços da camada de transporte;
                \item Multiplexação;
                \item Entrega confiável;
                \item UDP e TCP.
            \end{enumerate}
            
         \item[UNIDADE VI:] Camada de  aplicação;
	         \begin{enumerate}[itemsep=0em, topsep=0em]
            	\item Arquiteturas de aplicação: modelos cliente-servidor e P2P;
            	\item Características gerais;
            	\item HTTP, DNS, SMTP e FTP.
            \end{enumerate}
                                    
	\end{description}	
	
	\metodologia
	O conteúdo será apresentado através de aulas expositivas, com participação dos alunos e resolução de exercícios individualmente e em grupo. Para aplicar a teoria na prática serão feitas práticas de laboratório sobre temas presentes na ementa da disciplina.


	\avaliacao
	A avaliação é realizada de forma processual e cumulativa utilizando os instrumentos de avaliação especificados pelo Regulamento de Organização Didática em seu art. 94 \S~1$^\circ$, conforme for mais adequado. A frequência é obrigatória, respeitando os limites de ausência previstos em lei.
	
		
	\begin{bibbasica}
		\item BALL, Bill; DUFF, Hoyt. \textbf{Dominando Linux}: Red Hat e Fedora. Editora Pearson. ISBN 9788534615174. (BVU)
		\item KUROSE, James F. \textit{et al}. \textbf{Redes de computadores e a internet}: uma abordagem top-down. 6. ed. São Paulo, SP: Pearson Education do Brasil, 2013. 634 p. Bibliografia. ISBN 9788581436777. (BVU)
		\item NEMETH, Evi; SNYDER, Garth; HEIN, Trent R. \textbf{Manual completo do linux}: guia do administrador. 2. ed. São Paulo: Pearson Prentice Hall, 2007. 684 p. ISBN 9788576051121. (BVU)
		
		\item TANENBAUM, Andrew S. \textbf{Redes de computadores}. Tradução de Daniel Vieira. Revisão técnica de Benedito Isaías Lima Lopes. 5. ed. São Paulo: Pearson Prentice Hall, 2011. 582 p. Inclui bibliografia 9788576059240. (BVU) 

		
	\end{bibbasica}
	
	\begin{bibcomplementar}
	
		\item ALENCAR, Marcelo Sampaio de. \textbf{Engenharia de redes de computadores}. São Paulo: Érica, 2012. 286 p. Bibliografia. ISBN 9788536504117.
		\item COULOURIS, George; DOLLIMORE, Jean; KINDBERG, Tim. \textbf{Sistemas distribuídos}: conceitos e projeto. Tradução de João Eduardo Nóbrega Tortello. Revisão técnica de Alexandre da Silva Carissimi. 4. ed. Porto Alegre: Bookman, 2007. 784 p. Inclui bibliografia. ISBN 9788560031498.
		\item SOARES, Walace; FERNANDES, Gabriel. \textbf{Linux}: fundamentos. São Paulo: Érica, 2010. 206 p. Inclui referência e índice. ISBN 9788536503219.
		\item SOUSA, Lindeberg Barros de. \textbf{Projetos e implementação de redes}: fundamentos, soluções, arquiteturas e planejamento. 3. ed. , rev. São Paulo: Érica, 2013. 318 p. ISBN 9788536501666.		
		\item TANENBAUM, Andrew S.; STEEN, Maarten Van. \textbf{Sistemas distribuídos}: princípios e paradigmas. Tradução de Arlete Simille Marques. Revisão técnica de Wagner Luiz Zucchi. 2. ed. São Paulo: Pearson Prentice Hall, 2007. 402 p. Inclui bibliografia. ISBN 9788576051428. (BVU)

	\end{bibcomplementar}

\end{pud}




% BANCO DE DADOS

\begin{pud}

	\pudinfo{TSII.106}{Banco de Dados}{---}{40}{40}{4}{2$^\circ$ Semestre}
	
	
	\ementa
	Introdução a Banco de Dados. Instalação de um SGBD. Configuração de um SGBD. Conceitos Básicos: tabelas, campos e registros, chave primária, chave secundária, chave estrangeira. Modelagem: conceitual, modelo entidade-relacionamento, modelo relacional. Linguagem SQL Básica: DML, DDL, e programação. Normalização e dependência de dados. Linguagem SQL avançada. Projeto de um banco de dados. Gatilhos (\textit{Triggers}). Funções e procedimentos. \textit{Backup} e restauração.
	
	\objetivos
	\textbf{OBJETIVO GERAL}:
	Entender os conceitos básicos de um banco de dados relacional e como ele se relaciona com sistemas de informação WEB e \textit{Desktop}; Realizar consultas em um banco de dados existente a partir de um diagrama de Entidade/Relacionamento e um requisito formal; Construir um pequeno banco de dados a partir de uma lista de requisitos.	
	\newline\\	
	\textbf{OBJETIVOS ESPECÍFICOS}:
	\begin{itemize}
		\item Compreender os conceitos relacionados a sistemas gerenciadores de banco de dados;
		\item Conhecer técnicas de modelagem de dados;
		\item  Manipular bancos de dados por meio da linguagem de consulta SQL.
		\item  Implementar bancos de dados relacionais que ofereçam os serviços de seleção e manipulação de dados a usuários e aplicações, a partir do levantamento e análise dos requisitos de um ambiente;
		\item Construir requisições aos dados solicitadas por usuários e aplicações, utilizando instruções da Linguagem SQL.
	\end{itemize}
	
	
	\programa
	\begin{description}[itemsep=0em]
		\item[UNIDADE I:] Fundamentos de  Banco de Dados; 
	         \begin{enumerate}[itemsep=0em, topsep=0em]
                \item Bancos de dados;
                \item Sistemas Gerenciadores de Bancos de Dados;
                \item Sistemas de Banco de Dados;
                \item Projeto de Banco de Dados.
            \end{enumerate}
            
        \item[UNIDADE II:] Modelo Relacional;
	         \begin{enumerate}[itemsep=0em, topsep=0em]
                \item  Conceitos básicos (tabelas, campos e registros, chave primária, chave secundária e chave estrangeira);
				\item Restrições;
				\item Operações;
				\item  Normalização;
            \end{enumerate}
            
         \item[UNIDADE III:] Linguagem SQL;
	         \begin{enumerate}[itemsep=0em, topsep=0em]
				\item  Fundamentos da linguagem SQL;
				\item Definição de dados (DDL);
				\item Manipulação de dados (DML);
				\item  Consultas básicas e complexas;
				\item Programação.        
            \end{enumerate}
            
         \item[UNIDADE IV:]  Projeto de banco de dados  e Modelagem Conceitual;
	         \begin{enumerate}[itemsep=0em, topsep=0em]
				\item Fases do projeto de banco de dados;
				\item Mapeamento entre modelos entidade-relacionamento e relacional;
				\item Dependências funcionais, multivalorada e de junção.
            \end{enumerate}

         \item[UNIDADE V:] Projeto de banco de dados; 
	         \begin{enumerate}[itemsep=0em, topsep=0em]
				\item Modelagem conceitual em banco de dados;
				\item Modelo Entidade--Relacionamento;
				\item Diagramas Entidade--Relacionamento.      
            \end{enumerate}
            
         \item[UNIDADE VI:]  Banco de dados avançado.
	         \begin{enumerate}[itemsep=0em, topsep=0em]
				\item Funções e procedimentos;
				\item  Gatilhos (\textit{Triggers});
				\item \textit{Backup} e restauração.  
            \end{enumerate}
	\end{description}		
	
	
	\metodologia
	Aulas expositivas e interativas com uso de recursos audiovisuais. Atividades em grupo, exercícios de modelagem, codificação de consultas e atividades práticas no laboratório de informática utilizando ferramentas computacionais de modelagem de dados e SGBDs.


	\avaliacao
	A avaliação é realizada de forma processual e cumulativa utilizando os instrumentos de avaliação especificados pelo Regulamento de Organização Didática em seu art. 94 \S~1$^\circ$, conforme for mais adequado. A frequência é obrigatória, respeitando os limites de ausência previstos em lei.
	

	\begin{bibbasica}
			
        \item DATE, C. J. \textbf{Introdução a sistemas de banco de dados}. 8. ed. Rio de Janeiro: Elsevier, 2004. ISBN 9788535212730. 
		\item ELMASRI, Ramez. E.; NAVATHE, Shamkant B. \textbf{Sistemas de banco de dados}. 6. ed. São Paulo: Pearson Addison Wesley, 2011. ISBN 9788579360855. (BVU)
		\item SILBERSCHATZ, Abraham; KORTH, Henry; SUDARSHAN, S. \textbf{Sistema de banco de dados}. 6. ed. Rio de Janeiro: Elsevier, 2012. ISBN 9788535245356. 
		
	\end{bibbasica}
	
	\begin{bibcomplementar}
    
	    \item ALVES, William Pereira. \textbf{Bancos de dados}: teoria e desenvolvimento. São Paulo: Érica, 2009. 286 p. ISBN 9788536502557.
		\item GRAVES, Mark. \textbf{Projeto de banco de dados com XML}. São Paulo: Pearson Education do Brasil, 2003.  ISBN 9788534614719. (BVU)
		\item HOTKA, Dan. \textbf{Aprendendo Oracle 9i}. São Paulo: Pearson Education do Brasil, 2002. 454 p. ISBN 9788534613248. (BVU)
		\item LAUDON, Keneth C.; LAUDON, Jane P. \textbf{Sistemas de informação gerenciais}. 11. ed. São Paulo: Pearson Education do Brasil, 2014.  ISBN 9788543005850. (BVU)
		\item PUGA, Sandra; FRANÇA, Edson; GOYA, Milton. \textbf{Banco de dados}: implementação em SQL, PL/SQL e Oracle 11g. São Paulo: Pearson Education do Brasil, 2013. ISBN 9788581435329. (BVU)
		 
	\end{bibcomplementar}
		
\end{pud}

%%%%%%%%% NOVO 2o SEMESTRE


%% POO
\begin{pud}
	
	\pudinfo{TSII.201}{Programação Orientada a Objetos}{TSII.101}{40}{40}{4}{2$^\circ$ Semestre}
	

	\ementa
	Fundamentos do paradigma orientado a objetos (abstração, encapsulamento, classes, objetos, atributos, métodos e construtores), conceitos avançados de orientação a objetos (herança, polimorfismo, classes abstratas, interfaces e pacotes), metodologia de desenvolvimento orientada a objetos e aplicação dos conceitos de orientação a objetos através de uma linguagem de programação apropriada. \textit{Collections} (Estruturas de dados genéricas). Tratamento de exceções. Interface gráfica com o usuário.
	
	\objetivos
	Compreender o mundo real e usar a habilidade de abstração para mapeá-lo em classes e objetos a fim de construir programas que solucionem os mais variados problemas; Interpretar a necessidade do usuário e mapeá-la para diagramas UML que servirão de insumo para o projeto de sistemas orientado a objetos.
	
	\programa
	\begin{description}[itemsep=0em]
		\item[UNIDADE I:] Fundamentos da metodologia de desenvolvimento orientada a objetos; 
	         \begin{enumerate}[itemsep=0em, topsep=0em]
                \item Histórico das Linguagens de Programação;
                \item Introdução ao Paradigma Orientado a objetos.
            \end{enumerate}
            
        \item[UNIDADE II:] Fundamentos da Orientação a Objetos;
	         \begin{enumerate}[itemsep=0em, topsep=0em]
                \item Classes, objetos, atributos, métodos e construtores;
				\item Instanciação;
				\item Atributos estáticos;
				\item Modificadores de acesso;
				\item Cláusula de auto referência (\textit{this, self}).
				\item A Linguagem de Modelagem UML;
            \end{enumerate}
            
         \item[UNIDADE III:] Conceitos Avançados de Orientação a Objetos;
	         \begin{enumerate}[itemsep=0em, topsep=0em]
				\item Herança e polimorfismo;
				\item Operador de herança;
				\item Sobrecarga $\times$ sobrescrita de métodos;
				\item Classes abstratas;
				\item Interfaces;
				\item Tratamento de exceção;
				\item Coleções genéricas.      
            \end{enumerate}
            
         \item[UNIDADE IV:]  Projeto orientado a objetos;
	         \begin{enumerate}[itemsep=0em, topsep=0em]
				\item Fundamentos de projeto de software orientado a objetos;
				\item Diagrama de casos de uso e de classes;
				\item Interfaces gráficas de usuário;
				\item Testes unitários.
            \end{enumerate}
	\end{description}	
	
	\metodologia
	Aulas expositivas e interativas com uso de recursos audiovisuais.  Atividades em grupo e prática de codificação de algoritmos em linguagem de programação orientada a objetos. Atividades práticas no laboratório de codificação de programas.
	
	\avaliacao
	A avaliação é realizada de forma processual e cumulativa utilizando os instrumentos de avaliação especificados pelo Regulamento de Organização Didática em seu art. 94 \S~1$^\circ$, conforme for mais adequado. A frequência é obrigatória, respeitando os limites de ausência previstos em lei.
	
	

	\begin{bibbasica}
		\item BARNES, David J.; KÖLLING, Michael. \textbf{Programação orientada a objetos com Java}: uma introdução prática usando o BLUEJ. Tradução de Edson Furmankiewicz. Revisão técnica de João Luiz Silva Barbosa. 4. ed. São Paulo: Pearson Prentice Hall, 2009. 455 p. Inclui bibliografia. ISBN 9788576051879. 
		\item DEITEL, Paul; DEITEL, Harvey M. \textbf{Java}: como programar. 8. ed. Editora Pearson. Livro. 1178 p. ISBN 9788576055631. (BVU)
		\item HORSTMANN, Cay S.; CORNELL, Gary. \textbf{Core Java}: volume I - fundamentos. Tradução de Edson Furmankiewicz, Carlos Schafranski. Revisão técnica de Nivaldo Foresti. 8. ed. São Paulo: Pearson Prentice Hall, 2010. 383 p. (I). ISBN 9788576053576.
		
	\end{bibbasica}
	
	\begin{bibcomplementar}
    
        \item LARMAN, Craig. \textbf{Utilizando UML e padrões}: uma introdução à análise e ao projeto orientados a objetos e ao desenvolvimento iterativo. Tradução de Rosana T. Vaccare Braga. 3, ed. Porto Alegre: Bookman, 2007. 695 p. Inclui bibliografia. ISBN 9788560031528. 
		\item MCLAUGHLIN, Brett; POLLICE, Gary; WEST, David. \textbf{Use a cabeça}: análise  e projeto orientado ao objeto. Tradução de Betina Macêdo. Revisão técnica de Eduardo Velasco. Rio de Janeiro: Alta Books, 2007. 442 p. ISBN 9788576081456.
		\item MEDEIROS, Ernani Sales de. \textbf{Desenvolvendo software com UML 2.0}: definitivo. definitivo. São Paulo:  Pearson Makron Books, 2004. Livro. 288 p. ISBN 9788534615297.  (BVU)
		\item PAGE-JONES, Meilir. \textbf{Fundamentos do desenho orientado a objeto com UML}.  Editora Pearson. Livro. 488 p. ISBN 9788534612432.  (BVU)
		\item SANTOS, Rafael. \textbf{Introdução à programação orientada a objetos usando Java}. 2. ed. Rio de Janeiro: Elsevier, 2013. 313 p. (Campus SBC - Sociedade Brasileira de Computação). Inclui bibliografia e índice. ISBN 9788535274332.
	
	\end{bibcomplementar}
		
	
\end{pud}


%%%%%  Engenharia de Software

\begin{pud}
	\pudinfo{TSII.202}{Engenharia de Software}{---}{40}{40}{4}{2$^\circ$ Semestre}
	
	\ementa
	Definição de Engenharia de Software. Processos de desenvolvimento de software. Desenvolvimento ágil de software. Engenharia de requisitos. Projeto, desenvolvimento e evolução de sistemas. Documentação, testes e manutenção de software. Métricas e qualidade de software. Ambientes de desenvolvimento de software.
	
	\objetivos
	\textbf{OBJETIVO GERAL}:
	Entender os conceitos relacionados ao surgimento e a evolução da Engenharia de Software, suas técnicas e metodologias, sua aplicação e importância nas mais diversas áreas em que o desenvolvimento de software está presente.	
	\newline\\	
	\textbf{OBJETIVOS ESPECÍFICOS}:
	\begin{itemize}
		\item Assimilar o que é Engenharia de Software e qual a sua importância;
		\item Compreender questões profissionais e éticas relevantes para os engenheiros de software; 
		\item Conhecer as fases que compõem o processo de desenvolvimento de software;
		\item Analisar a importância do planejamento de projeto em todos os projetos de software;
		\item Aprender os principais modelos de processo a fim de saber quando e como aplicá-los; 
		\item Saber aplicar as metodologias de testes e qualidade de software;
		\item Avaliar os custos da evolução do software e a importância de utilização de boas práticas de desenvolvimento e padrões de projeto para uma evolução sustentável.
	\end{itemize}
	
	\programa
	\begin{description}[itemsep=0em]
		\item[UNIDADE I:] Visão geral de Engenharia de Software; 
	         \begin{enumerate}[itemsep=0em, topsep=0em]
                \item Conceitos e contextualização da Engenharia de Software;
				\item Princípios de Engenharia de Software;
                \item Ética na Engenharia de Software;
				\item Modelos de Software (genéricos e iterativos);
				\item Desenvolvimento ágil (\textit{Extreme Programming} e \textit{Scrum});
				\item Aspectos gerais das etapas de desenvolvimento de software.
            \end{enumerate}
            
        \item[UNIDADE II:] Requisitos de Software;
	         \begin{enumerate}[itemsep=0em, topsep=0em]
				\item Processo de engenharia de requisitos;
			    \item Técnicas de elicitação de requisitos; %(requisitos de usuário, requisitos de sistema, requisitos funcionais, não funcionais e de domínio);
			    \item Estudo de viabilidade;
			    \item Gerenciamento de requisitos;
			    \item Matriz de rastreabilidade.
            \end{enumerate}
            
         \item[UNIDADE III:] Gerência de projetos de software;
	         \begin{enumerate}[itemsep=0em, topsep=0em]
				\item Espectro da gestão;
				\item Planejamento e acompanhamento do projeto;
				\item Métricas de processo e projeto de software;
				\item Plano de projeto de software;
				\item Diagrama de barras (\textit{Gantt chart}).        
            \end{enumerate}
            
         \item[UNIDADE IV:]  Modelagem de sistemas;
	         \begin{enumerate}[itemsep=0em, topsep=0em]
				\item Linguagem de Modelagem Unificada (UML);
				\item Modelagem de casos de uso;
				\item Diagramas estruturais e comportamentais da UML.
            \end{enumerate}

         \item[UNIDADE V:] Verificação e Validação de Software;
	         \begin{enumerate}[itemsep=0em, topsep=0em]
				\item Planejamento de verificação e validação;
				\item Estratégias de teste de software, de release e de usuário;
				\item Testes automatizados com ferramentas de automação de testes;
				\item  Integração e entrega contínua.      
            \end{enumerate}
            
         \item[UNIDADE VI:]  Disponibilização, Evolução e Qualidade de Software.
	         \begin{enumerate}[itemsep=0em, topsep=0em]
				\item Disponibilização de software;
				\item Evolução e manutenção de software;
				\item Conceito de qualidade de software;
				\item Normas de qualidade do produto e do processo de software.  
            \end{enumerate}
	\end{description}
	
	\metodologia
	Aulas expositivas e interativas com uso de recursos audiovisuais. Atividades práticas relacionadas a técnicas e métodos atuais de Engenharia de Software. Além disso, atividades em grupo com pesquisa em artigos científicos da área de Engenharia de Software.
	
	\avaliacao
	A avaliação é realizada de forma processual e cumulativa utilizando os instrumentos de avaliação especificados pelo Regulamento de Organização Didática em seu art. 94 \S~1$^\circ$, conforme for mais adequado. A frequência é obrigatória, respeitando os limites de ausência previstos em lei.
	
	\begin{bibbasica}
			
        %\item PAULA FILHO, Wilson de Pádua. \textbf{Engenharia de Software}: fundamentos, métodos e padrões. 3. ed. Rio de Janeiro: LTC, 2009.
 		\item MEDEIROS, Ernani Sales de. \textbf{Desenvolvendo software com UML 2.0}: definitivo.definitivo. São Paulo:  Pearson Makron Books, 2004. Livro. 288 p. ISBN 9788534615297. (BVU) 
		
 		\item PRESSMAN, Roger S.; MAXIM, Bruce R. \textbf{Engenharia de software}: uma abordagem profissional.  Tradução de Ariovaldo Griesi, Mario Moro Fecchio. Revisão técnica de Reginaldo Arakaki, Renato Manzan de Andrade, Julio Arakaki. 7. ed. Porto Alegre: AMGH, 2011. 780 p. Inclui bibliografia. ISBN 9788563308337. 
 		\item SOMMERVILLE, Ian. \textbf{Engenharia de software}. Tradução de Kalinka Oliveira, Ivan Bosnic. Revisão técnica de Kechi Hirama. 9. ed. São Paulo: Pearson Prentice Hall, 2011. 529 p. Inclui bibliografia. ISBN 9788579361081. (BVU)
 		
	\end{bibbasica}
	
	\begin{bibcomplementar}
	
		
		
		\item ENGHOLM JÚNIOR, Hélio. \textbf{Engenharia de software na prática}. São Paulo: Novatec, 2010. 438 p. Inclui bibliografia. ISBN 9788575222171.
		\item LIMA, Adilson da Silva. \textbf{UML 2.3}: do requisito à solução. São Paulo: Érica, 2011. 368 p. Inclui bibliografia e índice. ISBN 9788536503776.
		\item MAGELA, Rogério. \textbf{Engenharia de software aplicada}: fundamentos. Rio de Janeiro: Alta Books, 2006. 418 p. Inclui bibliografia. ISBN 8576081237.
		\item PFLEEGER, Shari Lawrence. \textbf{Engenharia de Software}: teoria e prática. 2. ed. São Paulo: Prentice Hall, 2004. Livro. 560 p.  ISBN 8587918311 (BVU)
		\item  VAZQUEZ, C	arlos Eduardo; SIM\~OES, Guilherme Siqueira. \textbf{Engenharia de Requisitos}: software orientado ao negócio. Editora Brasport, 2016.  294 p. ISBN 9788574527963. (BVU)
		

% 	    \item BOOCH, Grady; RUMBAUGH, James; JACOBSON, Ivar. \textbf{UML}: guia do usuário.  2. ed. Rio de Janeiro: Elsevier, 2012.
% 		\item BOURQUE, Pierre; FAIRLEY, Richard E. \textbf{SWEBOK v3.0}: guide to the software engineering body of knowledge. IEEE Computer Society Press, 2014. Disponível em: \url{www.computer.org/web/swebok/v3}. Acesso em: 16 ago. 2018.
% 		\item COHN, Mike. \textbf{Desenvolvimento de software com SCRUM}: aplicando métodos ágeis com sucesso. Porto Alegre: Bookman, 2011.
% 		\item SHORE, J.; WARDEN, S. \textbf{A arte do desenvolvimento ágil}. Rio de Janeiro: Alta Books, 2008.
% 		\item WAZLAWICK, Raul Sidney. \textbf{Engenharia de software}: conceitos e práticas. Rio de Janeiro: Elsevier, 2013.
% 		 
	\end{bibcomplementar}
\end{pud}



% ETICA

\begin{pud}

	\pudinfo{TSII.203}{Ética e Responsabilidade Socioambiental}{--}{20}{4}{2}{1$^\circ$ Semestre}
	
	
	\ementa 
	Explora temáticas relacionadas às questões étnico-raciais, história e
	cultura afro-brasileira e indígena, bem como educação ambiental. Trabalha o
	desenvolvimento de projetos para resolução de problemas que envolvam
	as temáticas em questão de forma integradora. Busca desenvolver uma cultura
	científica interdisciplinar.
	
	\objetivos
	\textbf{OBJETIVO GERAL}:
	Possibilitar o desenvolvimento de aprendizagens no âmbito da produção científica consolidada na forma de projetos integradores que envolvam saberes em uma perspectiva interdisciplinar envolvendo temas como tecnologia, questões étnico-raciais, indígenas e educação ambiental.	
	\newline\\	
	\textbf{OBJETIVOS ESPECÍFICOS}:
	\begin{itemize}
		\item Discutir e relacionar os conceitos de tecnologia, cultura e sociedade;
        \item Discutir os conceitos identidade, identidade negra, raça, etnia, racismo, etnocentrismo, preconceito racial, discriminação racial, democracia racial; políticas de ações afirmativas; 
        \item Apontar acontecimentos que demonstrem o protagonismo negro e indígena no mercado de trabalho;
        \item Pensar a pesquisa e o desenvolvimento de projeto a partir de uma perspectiva interdisciplinar;
        \item Desenvolver propostas de pesquisa que contemplem soluções científicas em que dialoguem tecnologia, questões étnico-raciais, indígena e educação ambiental.
	\end{itemize}
	
	
	\programa
	\begin{description}[itemsep=0em]
		\item[UNIDADE I:] Tecnologia e sociedade;	 
	         \begin{enumerate}[itemsep=0em, topsep=0em]
				\item Conceito de Tecnologia;
				\item Sociedade na Perspectiva Contemporânea.
            \end{enumerate}
            
        \item[UNIDADE II:] Questões etino-racial e indígena;
	         \begin{enumerate}[itemsep=0em, topsep=0em]
				\item Racismo Estrutural, Identidade, Etnocentrismo, Preconceito racial e Discriminação racial;
                \item Políticas de Ações Afirmativas;
                \item Cultura afro-brasileira e indígena;
                \item Empreendedorismo negro;
                \item Protagonismo negro e indígena no mercado de trabalho.
            \end{enumerate}
            
         \item[UNIDADE III:]  Educação Ambiental e Questões Sociais;
	         \begin{enumerate}[itemsep=0em, topsep=0em]
				\item Meio Ambiente e Educação;
				\item Problemas Ambientais Contemporâneos.       
            \end{enumerate}
            
         \item[UNIDADE IV:] Projeto integrador.
	         \begin{enumerate}[itemsep=0em, topsep=0em]
                \item Conceito de Ciência;
                \item Tipos de Pesquisa e Interdisciplinaridade;
                \item Estrutura do Projeto de Pesquisa Integrador.
            \end{enumerate}
	\end{description}
	
	
	\metodologia
	A metodologia aborda uma postura diversificada contemplando a seguintes ações: (i) Aulas expositivas do tipo dialogadas; (ii) Leituras e discussão de textos de forma coletiva; e (iii) Rodas de conversa sobre os conteúdos estudados nas demais disciplina a fim de percebê-los como elementos norteadores para a definição dos temas de pesquisa.
	A carga horária destinada às atividades práticas será contemplada nos momentos de orientação e também de escrita do projeto, bem como sua socialização em sala de aula.


	\avaliacao
	A avaliação da aprendizagem deve considerar inicialmente as discussões em sala de aula, bem como os textos escritos durante a disciplina. Também, deve-se avaliar a consolidação dos projetos integradores com temáticas que envolvam tecnologia e os demais temas como questões étnico- raciais e indígenas e educação ambiental.
	
	

	\begin{bibbasica}
			
        \item AFONSO, Germano. B.; CREMONEZE, Cristina; BUENO, Luiz. (Orgs). \textbf{Ensino de História e Cultura Indígenas}. Curitiba: InterSaberes, 2016. (BVU)
		\item BOSI, Alfredo. \textbf{Dialética da colonização}. 4. ed. São Paulo: Companhia das Letras, 2016. 420 p. ISBN 978-85-7164-276-8. %10 Exs. 981 B741d
		\item CERVO, Amado Luiz; BERVIAN, Pedro Alcino; SILVA, Roberto da. \textbf{Metodologia científica}. 6. ed. São Paulo: Pearson Prentice Hall, 2007. 159 p. Inclui bibliografia. ISBN 8576050471. (BVU) %CAMPUS TAUÁ: 7 ex. 001.42 C419m
		\item GIL, Antônio Carlos. \textbf{Métodos e técnicas de pesquisa social}. 6.ed.  São Paulo: Atlas, 2016. 200 p. ISBN 978-85-224-5142-5. %CAMPUS TAUÁ: 30 ex. 001.42 G463m
		\item MARCONI, Marina de Andrade; LAKATOS, Eva Maria. \textbf{Fundamentos de metodologia científica}. 7. ed. São Paulo: Atlas, 2010. 297 p. Inclui bibliografia. ISBN 978852245758. %CAMPUS TAUÁ: 8 ex. 001.42 M321f
		\item PINOTTI, Rafael. \textbf{Educação ambiental para o século XXI}: no Brasil e no mundo. 2. ed. São Paulo: Blucher, 2016. (BVU)
		\item SABBAG, Paulo Yazigi. {\bfseries Gerenciamento de projetos e empreendedorismo}. 2. ed. São Paulo: Saraiva, 2013. 226 p. Inclui bibliografia. ISBN 9788502204447. %CAMPUS TAUÁ: 7 ex. 658.404 S114g
		\item SEVERINO, Antônio Joaquim. {\bfseries Metodologia do trabalho científico}. 23. ed. rev. e atual São Paulo: Cortez, 2012. 304 p. Inclui bibliografia. ISBN 9788524913112. %CAMPUS TAUÁ: 7 ex. 001.42 S498m
	 
	\end{bibbasica}
	
	\begin{bibcomplementar}
        	\item AMARO, Sarita. \textbf{Racismo, igualdade racial e políticas de ações afirmativas no Brasil}.  Porto Alegre: EDIPUCRS, 2015. (BVU)
		
 			\item ARAÚJO, Ulisses F.  {\bfseries Temas transversais, pedagogia de projetos e mudanças na educação}.  São Paulo: Summus, 2014. (BVU)
 			\item FAZENDA, Ivani C. A.; TAVARES, Dirce E.; GODOY, Hermínia P. \textbf{Interdisciplinaridade na pesquisa científica}. [livro eletrônico].  Campinas, SP: Papirus Editora, 2017.  ISBN 9788544902776. (BVU)
		
		%	\item CERVO, Amado Luiz; BERVIAN, Pedro Alcino; SILVA, Roberto da.  {\bfseries Metodologia científica}. 6. ed. São Paulo: Pearson Prentice Hall, 2007. 159 p. Inclui bibliografia. ISBN 8576050471. % CAMPUS TAUÁ: 7 ex. 001.42 C419m
%			\item FHILIPPI JR., Arlindo (Coord).  {\bfseries Direitos humanos e meio ambiente}: minorias ambientais. [recurso eletrônico]. Barieri, SP: Manole, 2017. %(BVU)
			\item MACHADO, Anna Rachel; LOUSADA, Eliane; ABREU-TARDELLI, Lília Santos.  {\bfseries Planejar gêneros acadêmicos}: escrita científica, texto acadêmico, diário de pesquisa, metodologia. São Paulo: Parábola, 2005. 116 p. (Leitura e produção de textos técnicos e acadêmicos ; 3). Inclui bibliografia. ISBN 9788588456433. %CAMPUS TAUÁ: 3 ex. 001.42 M149p
			\item MATTAR, João.  {\bfseries Metodologia científica na era da informática}. 3. ed. rev. e atual São Paulo: Saraiva, 2008. 308 p. Inclui bibliografia. ISBN 9788502064478.
			\item MATTOS, João Roberto Loureiro de; GUIMARÃES, Leonam dos Santos.  {\bfseries Gestão da tecnologia e inovação}: uma abordagem prática. 2. ed. e atual São Paulo: Saraiva, 2012. 433 p. Inclui referência. ISBN 9788502178946. %CAMPUS TAUÁ: 7 ex. 658.4 M389g
			
			%\item ULISSES F. ARAÚJO. \textbf{Temas transversais, pedagogia de projetos e mudanças na educação}. Summus Editorial. Livro. 120 p. ISBN 9788532309679. 
	\end{bibcomplementar}
		
	
\end{pud}


%%%%%%%% EMPREENDEDORISMO

\begin{pud}
	\pudinfo{TSII.204}{Empreendedorismo}{---}{36}{4}{2}{1$^\circ$ Semestre}
	
	
	\ementa
	Introdução ao ``Mundo dos Negócios''; Industria 4.0; Economia criativa $\times$ Economia tradicional; Conceitos de empreendedorismo e inovação; Tipos de empreendedorismo; Atitude empreendedora; Protagonismo empreendedor (Mulheres, Negros e outras minorias); Descoberta $\times$ Invenção $\times$ Inovação; Tipos de inovação e \textit{Open innovation}; Ideias $\times$ Oportunidades: como identificar oportunidades de negócios; \textit{Startup}: Conceito e tipos de \textit{Startup}; Estágios de um \textit{Startup}; Ecossistema Empreendedor; Metodologias de Modelagem de Negócios: \textit{Lean Startup, Business Model Canvas, Design Thinking} e Plano de Negócios; Tipos de assessorias: incubadoras, aceleradoras, \textit{franchising}, mentoria, investidor anjo e capitalista de risco; Fontes de financiamentos: Recursos próprios, Subvenções, \textit{Crowdfunding}, Aceleradoras e Fundos de Investimentos; Arranjos empresariais: Arranjos produtivos locais (APL), \textit{clusters} e rede de empresas; Futuro do perfil empreendedor: Competências, Habilidades, e Atitudes. 
	
	\objetivos
		Propiciar ao discente o conhecimento teórico das competências, habilidades e atitudes empreendedoras; 
		Apresentar os conceitos e tipos de empreendedorismo; atitudes empreendedoras e inovação;
		Diferenciar ideias/oportunidade e economia tradicional/criativa;
		Conceituar um Startup;
		Identificar um Startup;
		Conhecer a aplicação das ferramentas empreendedoras;
		Conhecer os tipos de assessoria, financiamentos e arranjos empresariais; 
		Compreender o perfil do empreendedor no futuro.
        \newline
        \textbf{OBS}: a aplicação do conhecimento teórico será desenvolvido no Projeto Integrador I e II.
	
	
	\programa
	\begin{enumerate}[itemsep=0em]
		\item Introdução ao  ``Mundo dos Negócios'';
		\item Industria 4.0;
		\item Economia criativa $\times$ Economia tradicional; 
		\item Conceitos de empreendedorismo e inovação;
		\item Tipos de empreendedorismo;
		\item Atitude empreendedora;
		\item Protagonismo empreendedor (Mulheres, Negros e outras minorias);
		\item Descoberta $\times$ Invenção $\times$ Inovação;
		\item Tipos de inovação e \textit{Open innovation}; 
		\item Ideias $\times$ Oportunidades: como identificar oportunidades de negócios;
		\item \textit{Startup}: Conceito e tipos de \textit{Startup}, Estágios de um \textit{Startup} e Ecossistema Empreendedor;
		\item Metodologias de Modelagem de Negócios: \textit{Lean Startup, Business Model Canvas, Design Thinking} e Plano de Negócios;
		\item Tipos de assessoria: Incubadoras, Aceleradoras, \textit{Franchising}, Mentoria, Investidor anjo e Capitalista de risco;
		\item Fontes de financiamentos: Recursos próprios, Subvenções, \textit{Crowdfunding}, Aceleradoras e Fundos de Investimentos; 
		\item Arranjos empresariais: Arranjos produtivos locais (APL), \textit{clusters} e rede de empresas;
		\item Futuro do perfil empreendedor: competências e habilidades.
		 
	\end{enumerate}
	
	
	\metodologia
	Aulas teóricas: expositivo-dialogadas com aplicação e resolução de exercícios, estudos dirigidos, seminários, vídeos e dinâmicas de grupo. Aulas práticas: realizadas em jogos simulados, laboratório, visitas técnicas e/ou participações em eventos. Recursos: quadro branco, cartolina, pincéis, post-it, lousa digital, data-show, aparelho de som, computador pessoal, smartphone, Internet, e-mail; redes sociais, Ambiente Virtual de Aprendizagem (AVA) e outros.
	
	\avaliacao
	Verificação de conhecimentos através de avaliação presencial, avaliação a distância desenvolvidas em Ambiente Virtual de Aprendizagem empregando a metodologia de avaliação disponível no Google Sala de Aula e auto avaliação permitindo ao aluno saber seu desempenho.
	A avaliação será desenvolvida nas seguintes formas:
		\begin{description}[itemsep=0em]
            \item[$\bullet$ Diagnóstica --] levantamento dos conhecimentos prévio dos alunos;
            \item[$\bullet$ Continuada --] análise de todo o processo de ensino-aprendizagem observando a participação individual e em grupo, o envolvimento nas atividades, o desenvolvimento dos conteúdos e o nível de percepção apresentado, isto é, o olhar não apressado que consegue descobrir detalhes, estabelecer comparações e conexões com o dia-a-dia, a condição humana, enfim, a própria vida;            
        \end{description}
        
        Tipos de verificação do conhecimento construído durante as aula: 
	    \begin{description}[itemsep=0em]
            \item[$\bullet$ Escrita,] através de questionário individual e/ou equipe;
            \item[$\bullet$ Oral,] através de apresentação individual e/ou equipe;
            
        \end{description}
		Os recursos avaliativos serão utilizados com base  no art. 94 \S~1$^\circ$ alínea de I a XV do Regulamento de Organização Didática.

	\begin{bibbasica}
		\item DEGEN, Ronald Jean. \textbf{O Empreendedor}: fundamentos da iniciativa empresarial. 8. ed. Editora Pearson. Livro. 384 p. ISBN 9788534602174.  (BVU) % Disponível em: <http://ifce.bv3.digitalpages.com.br/users/publications/9788534602174>. Acesso em: 31 mar. 2020.
        
		%\item DORNELAS, José Carlos Assis. \textbf{Empreendedorismo na prática}: mitos e verdades do empreendedor de sucesso. 3.ed. Rio de Janeiro: LTC, 2015. 141 p.
		\item MAXIMIANO, Antonio Cesar Amaru. \textbf{Empreendedorismo}. Editora Pearson. Livro. 186 p. ISBN 9788564574342. (BVU)
		
		\item SERTEK, Paulo. \textbf{Empreendedorismo}.  Editora Intersaberes. Livro. 240 p.  ISBN 9788565704199. (BVU) %Disponível em: <http://ifcefortaleza.bv3.digitalpages.com.br/users/publications/9788565704199/pages/-2>. Acesso em: 31 mar. 2020.
         
		
	\end{bibbasica}
	
	\begin{bibcomplementar}
		\item ACADEMIA PEARSON. \textbf{Criatividade e Inovação}. São Paulo: Pearson Prentice Hall, 2011. Livro. 150 p. ISBN 9788576058847. (BVU)
        \item BERNARDI, Luiz Antonio. \textbf{Manual de empreendedorismo e gestão}: fundamentos, estratégias e dinâmicas. São Paulo: Atlas, 2012. 314 p. ISBN 9788522433384. 
		
        \item DEGEN, Ronald Jean. \textbf{O Empreendedor}: empreender como opção de carreira. São Paulo: Pearson Prentice Hall, 2009. 440 p. ISBN 9788576052050.  (BVU)
		\item DORNELAS, José Carlos Assis. \textbf{Empreendedorismo na prática}:  mitos e verdades do empreendedor de sucesso. 3.ed. Rio de Janeiro: LTC, 2015. 141 p. ISBN 9788521627920. 
       
         \item MAXIMIANO, Antonio Cesar Amaru. \textbf{Administração para empreendedores}: fundamentos da criação e gestão de novos negócios. 2. ed. Pearson.  258 p. ISBN 9788576058762.  (BVU) %Disponível em: <http://ifce.bv3.digitalpages.com.br/users/publications/9788576058762>. Acesso em: 31 mar. 2020.
% 		%\item MORAIS, Roberto Souza de. \textbf{O profissional do futuro}: uma visão empreendedora. Manole. E-book. 156 p. %(BVU)%  Disponível em: <http://ifce.bv3.digitalpages.com.br/users/publications/9788578680978>. Acesso em: 31 mar. 2020.
%         \item OSTERWALDER, Alexandrer; PIGNEUR, Yves. Business \textbf{Model Generation}: A Handbook for Visionaries, Game Changers, and Challengers. John Wiley \& Sons, 2013. Disponível em: \url{https://www.strategyzer.com/books/business-model-generation}. Acesso em: 31 mar. 2020. 
%         \item OSTERWALDER, Alexandrer \textit{et al}. \textbf{Testing Business Ideas}: A Field Guide for Rapid Experimentation. John Wiley \& Sons, 2019. Disponível em: \url{https://www.strategyzer.com/books/testing-business-ideas-david-j-bland}. Acesso em: 31 mar. 2020.
%         \item OSTERWALDER, Alexandrer \textit{et al}. \textbf{Value Proposition Design}: How to Create Products and Services Customers Want. John Wiley \& Sons, 2015. Disponível em: \url{https://www.strategyzer.com/books/value-proposition-design}. Acesso em: 31 mar. 2020. 
%         \item SILVA, Lacy de Oliveira; GITAHY, Yuri. \textbf{Disciplina de empreendedorismo e inovação}: manual do estudante. Brasília: Sebrae, 2016. Disponível em: \url{https://drive.google.com/file/d/0B5ytz8zSeF7ZUFRVZzlKSUdNUWpZZnpfR2Q5R1FueTNodlRF/view?usp=sharing}. Acesso em: 31 mar. 2020.	

	\end{bibcomplementar}
	
		
\end{pud}




%%% ESTRUTURA DE DADOS

\begin{pud}
	
	\pudinfo{TSII.205}{Estrutura de Dados}{TSII.102}{40}{40}{4}{2$^\circ$ Semestre}
	
	\ementa
	Tipos abstratos de dados. Listas lineares e suas generalizações: listas ordenadas, listas encadeadas, pilhas e filas. Variáveis dinâmicas. Ordenação e Busca. Árvores.
	
	\objetivo
	Empregar estruturas de dados adequadas para o desenvolvimento de sistemas de software.
	\begin{itemize}
	  \item Definir e diferenciar as estruturas de dados genéricas fundamentais, tais como filas, pilhas, listas encadeadas e grafos;
	  \item Manipular estruturas de dados através do emprego de algoritmos;
	  \item Selecionar e construir estruturas de dados adequadas para aplicações específicas;
	  \item Construir algoritmos de ordenação e busca de acordo com a estratégia mais adequada.
	\end{itemize}
	
	
	\programa
	
	\begin{description}[itemsep=0em]
		\item[UNIDADE I:] Tipos Abstratos de Dados; 
         \begin{enumerate}[itemsep=0em, topsep=0em]
           \item Conceitos preliminares;
           \item Definição de tipos abstratos de dados;
           \item Alocação dinâmica de memória.
         \end{enumerate}
         
         \item[UNIDADE II:] Listas encadeadas; 
         \begin{enumerate}[itemsep=0em, topsep=0em]
           \item Listas estáticas e dinâmicas;
           \item Listas simples;
           \item Listas duplamente encadeadas; 
           \item Listas circulares;
           \item Operações sobre listas.
         \end{enumerate}
         
         \item[UNIDADE III:] Pilhas; 
         \begin{enumerate}[itemsep=0em, topsep=0em]
           \item Pilhas estáticas e dinâmicas;
           \item Operações sobre pilhas.
         \end{enumerate}
         
         \item[UNIDADE IV:] Filas; 
         \begin{enumerate}[itemsep=0em, topsep=0em]
           \item Filas estáticas e dinâmicas;
           \item Operações sobre filas.
         \end{enumerate}
         
         \item[UNIDADE V:] Árvores e suas generalizações; 
         \begin{enumerate}[itemsep=0em, topsep=0em]
           \item Conceitos, implementação e operações sobre árvores;
           \item Árvores Binárias;
           \item Árvores Balanceadas -- AVL, rubro-negra e árvores-B.
         \end{enumerate}
         
         \item[UNIDADE VI:] Tipos Abstratos de Dados; 
         \begin{enumerate}[itemsep=0em, topsep=0em]
           \item Bubble Sort;
           \item Selection Sort;
           \item Insertion Sort;
           \item Merge Sort;
           \item Quick Sort;
           \item Árvore binária de busca;
           \item Algoritmos de busca.
         \end{enumerate}
	\end{description}
	
	\metodologia
	Aulas expositivas e interativas com uso de recursos audiovisuais. Atividades
práticas em laboratório, realização de trabalhos em equipe e individuais. Pesquisas
em artigos científicos e repositórios de códigos-fontes de programas de computador.

	\recursos
	Data-show, pincel e quadro branco, laboratório de informática, computadores,
softwares para apoio em classe e extraclasse e softwares para programação.

	\avaliacao
	A avaliação é realizada de forma sistemática, periódica e cumulativa utilizando os
instrumentos de avaliação especificados pelo Regulamento de Organização
Didática (ROD) em seu art. 94 § 1º, conforme for mais adequado. A frequência é
obrigatória, respeitando os limites de ausência previstos em lei.
	\naopresencial
	
	
	\begin{bibbasica}
		\item CELES, Waldemar; CERQUEIRA, Renato; RANGEL, José Lucas. \textbf{Introdução a
estruturas de dados}: com técnicas de programação em C. Rio de Janeiro:
Elsevier, 2016. 394 p. ISBN 9788535283457.
		\item CORMEN, Thomas H.; LEISERSON, Charles E.; RIVEST, Ronald L.; CLIFFORD,
Stein. \textbf{Algoritmos}: teoria e prática. Rio de Janeiro: Campus, 2012. 926 p. ISBN
9788535236996.
		\item TENEMBAUM, Aaron M.; LANGSAM, Yedidyah; AUGENSTEIN, Moshe J.
\textbf{Estruturas de dados usando C}. São Paulo: Pearson Makron Books, 1995. 884 p.
ISBN 9788534603480.
	\end{bibbasica}
	
	
	\begin{bibcomplementar}
		\item ASCENCIO, Ana Fernanda Gomes; ARAÚJO, Graziela Santos de. \textbf{Estruturas de
Dados}: algoritmos, análise da complexidade e implementações em Java e C/C++.
São Paulo: Pearson Prentice Hall, 2011.432 p. ISBN 9788576058816.
		\item BACKES, André Ricardo. \textbf{Estrutura de dados descomplicada}: em linguagem C.
1. ed. Rio de Janeiro: Elsevier, 2016.
		\item LORENZI, Fabiana; MATTOS, Patrícia Noll de; CARVALHO, Tanisi Pereira de.
\textbf{Estruturas de Dados}. São Paulo: Thompson, 2007.175 p. ISBN 9788577803811.
		\item PUGA, Sandra. RISSETI, Gerson. \textbf{Lógica de programação e estrutura de
dados, com aplicações em Java}. 2. ed. São Paulo: Pearson Prentice Hall,
2009.262 p. ISBN 978857605207.
\item SCHILDT, Herbert. \textbf{C}: completo e total  3. ed. São Paulo: Pearson Makron Books,
1997. 827 p. ISBN 9788534605953.
	\end{bibcomplementar}
	
\end{pud}


%%% DEV WEB I
\begin{pud}
	
	\pudinfo{TSII.206}{Desenvolvimento WEB I}{TSII.102, TSII.104}{40}{40}{4}{2$^\circ$ Semestre}
	
	\ementa
	Linguagem de desenvolvimento \textit{back-end}: Introdução, Condicionais e Repetição, Funções, Formulários, Arquivos, Manutenção de Estado (Sessions e Cookies) e Orientação a Objetos; Aplicação Web: Dinâmica, Gerenciamento de sessão, Controle de Cache, Protocolos de comunicação;  Acesso a banco de dados em sistemas Web (\textit{back-end}): Conexão com Sistema de Gerenciamento de Banco de Dados; CRUD e RESTful API. Ciclo de desenvolvimento versionado: Controle de versões, Ferramentas de versionamento, \textit{Branching} e \textit{tracking}, e Correção de erros.
	
	\objetivos
	\textbf{OBJETIVO GERAL}:
	Construir páginas Web interativas observando as boas práticas de construção e formatação de seus elementos, além de aprender sobre a integração destes elementos com serviços externos a partir do auxílio de linguagem de programação WEB.	
	\newline\\	
	\textbf{OBJETIVOS ESPECÍFICOS}:
	\begin{itemize}
		
		\item Conhecer os diversos elementos da construção de interfaces WEB de modo a fazer uso de tais elementos de forma eficiente na construção de projetos;
		\item Construir interfaces Web utilizando modelos e métodos consolidados pelo mercado e indústria;
		\item Expor o que há de novo na área de desenvolvimento WEB, tanto na construção quanto na manutenção e progressão de softwares online, de modo a acentuar a progressão profissional do discente.

	\end{itemize}
	
	\programa
	\begin{description}[itemsep=0em]
		\item[UNIDADE I:] Linguagem de desenvolvimento \textit{back-end}; 
	         \begin{enumerate}[itemsep=0em, topsep=0em]
                \item Introdução;
				\item Condicionais e estruturas de repetição;
				\item Funções;
				\item Formulários e requisições;
				\item Arquivos;
				\item Manutenção de Estado (\textit{Sessions} e \textit{Cookies}).
            \end{enumerate}
            
        \item[UNIDADE II:] Aplicação WEB;
	         \begin{enumerate}[itemsep=0em, topsep=0em]
				\item Gerenciamento de sessão;
				\item Controle de Cache;
				\item Controle de acesso (autenticação e autorização);
				\item Protocolos de comunicação.
            \end{enumerate}
            
         \item[UNIDADE III:] Persistência e manipulação de dados em servidores \textit{back-end};
	         \begin{enumerate}[itemsep=0em, topsep=0em]
				\item Interação com Banco de Dados;
				\item Formatos de transporte de dados (JSON, XML, texto plano);
				\item Interface de Programação de Aplicações (API) para WEB;
				\item CRUD e RESTful API.        
            \end{enumerate}
            
         \item[UNIDADE IV:]  Ciclo de desenvolvimento versionado;
	         \begin{enumerate}[itemsep=0em, topsep=0em]
				\item Controle de versões;
                \item Ferramentas de versionamento;
                \item \textit{Branching} e \textit{tracking};
                \item Correção de erros.
            \end{enumerate}

	\end{description}
	
	
	\metodologia
	Aulas expositivas e interativas com uso de recursos audiovisuais. Atividades práticas relacionadas a técnicas e métodos atuais de Desenvolvimento WEB. Além disso, trabalhos individuais e em grupo, utilização de laboratório com exercícios práticos que possam auxiliar o treinamento e desenvolvimento de aplicações para WEB, envolvendo todos os aspectos aprendidos durante a disciplina e o curso.
	
	\avaliacao
	A avaliação é realizada de forma processual e cumulativa utilizando os instrumentos de avaliação especificados pelo Regulamento de Organização Didática em seu art. 94 \S~1$^\circ$, conforme for mais adequado. A frequência é obrigatória, respeitando os limites de ausência previstos em lei.
	
	\begin{bibbasica}
			
        \item DALL'OGLIO, Pablo. \textbf{PHP}: programando com orientação a objetos. 2. ed. São Paulo: Novatec Editora, 2009.  ISBN 9788575222003.
		\item DEITEL, Paul J.; DEITEL, Harvey M.  \textbf{Ajax, Rich Internet Applications e Desenvolvimento Web para Programadores}. Editora Pearson. 776 p. ISBN 9788576051619. (BVU)
		%\item OLIVIERO, Carlos A. J. \textbf{Faça um site PHP 5.2 com MySQL 5.0}: comércio eletrônico orientado por projeto. São Paulo: Érica, 2010.
		\item SOARES, Walace. \textbf{Crie um framework para sistemas web e com PHP 5 e Ajax}. São Paulo: Érica, 2009.  ISBN 9788536502373.
		
	\end{bibbasica}
	
	\begin{bibcomplementar}
    
	    
		\item FLATSCHART, F\'abio. \textbf{HTML 5}: embarque Imediato. Editora Brasport. 256 p. ISBN 9788574525778. (BVU)
		\item LEMAY, Laura; COLBURN, Rafe; TYLER, Denise. \textbf{Aprenda a Criar P\'aginas Web com HTML e XHTML em 21 Dias}. São Paulo: Pearson Education do Brasil, 2002.  1158 p. ISBN 9788534614283. (BVU)
 		\item MARINHO, Antônio Lopes (Org.). \textbf{Desenvolvimento de aplicações para Internet}. São Paulo: Pearson Education do Brasil, 2016. Livro. 139 p. ISBN 9788543020112. (BVU)
 		
 		\item PUGA, Sandra; FRANCA, Edson; GOYA, Milton. \textbf{Banco de dados}: Implementa\c{c}\~ao em SQL, PL/SQL e Oracle 11g.  São Paulo: Pearson Education do Brasil, 2013. 356 p.  ISBN 9788581435329. (BVU)
 		\item SEGURADO, Valquiria Santos (Org.). \textbf{Projeto de interface com o usuário}. São Paulo: Pearson Education do Brasil, 2015. Livro. 195 p. ISBN 9788543017303. (BVU) 
 		%\item SILVA, Maur\'icio Samy. \textbf{JavaScript}: guia do programador. S\~ao Paulo: Novatec, 2010. 604 p. Inclui bibliografia e \'indice. ISBN 9788575222485.  
		 
	\end{bibcomplementar}
	

\end{pud}





%%%% 3o SEMESTRE

%%%% PROGRAMAÇÃO PARA DISPOSITIVOS MÓVEIS

\begin{pud}

	\pudinfo{TSII.301}{Programação para Dispositivos Móveis}{TSII.106, TSII.201}{40}{40}{4}{3$^\circ$ Semestre}


	\ementa
	Histórico do desenvolvimento dos aplicativos móveis; Interfaces gráficas; Eventos de aplicações; Aplicações multimídia; Comunicação; Persistência de dados; Mapas e Geolocalização; Desenvolvimento prático de um sistema para dispositivos móveis; Fundamentos de teste de software; Ferramentas e estratégias de testes para aplicativos móveis.
	
	\objetivos
	\textbf{OBJETIVO GERAL}:
	Aprender a construir os mais variados aplicativos para dispositivos móveis, com foco em \textit{smartphones} e \textit{tablets}, de modo a possibilitar ao aluno a oportunidade de exercitar conceitos de Programação Orientada a Objetos e desenvolvimento ágil, aplicando-os a dispositivos móveis; Desenvolver casos de testes para as diversas situações e fases do desenvolvimento do aplicativo, de tal forma que o aluno possa aplicar os conhecimentos e ter uma visão geral da área de verificação, validação e teste de software no âmbito do desenvolvimento de aplicativos.	
	\newline\\	
	\textbf{OBJETIVOS ESPECÍFICOS}:
	\begin{itemize}
		
		\item Identificar características de potenciais aplicativos;
		\item Arquitetar aplicações para dispositivos móveis; 
		\item Implementar aplicações para dispositivos móveis; 
		\item Aplicar técnicas de desenvolvimento de softwares em dispositivos móveis. 

	\end{itemize}
	
	
	\programa
	\begin{description}[itemsep=0em]
		\item[UNIDADE I:] Introdução ao desenvolvimento para dispositivos móveis; 
	         \begin{enumerate}[itemsep=0em, topsep=0em]
				\item Evolução dos dispositivos móveis;
				\item Características dos dispositivos móveis;
				\item Arquiteturas de aplicação móvel;
				\item Infraestrutura móvel.
            \end{enumerate}
            
        \item[UNIDADE II:] Tratamento de eventos (interações do usuário);
	         \begin{enumerate}[itemsep=0em, topsep=0em]
				\item Padrões de projetos mais comuns para tratamento de eventos;
				\item Componentes gráficos, eventos relacionados e formas de tratamento.
            \end{enumerate}

		\item[UNIDADE III:] Aplicações multimídia; 
	         \begin{enumerate}[itemsep=0em, topsep=0em]
				\item Armazenamento e processamento de dados multimídia;
				\item Tratamento de eventos relacionados a imagens e sons.
            \end{enumerate}
            
        \item[UNIDADE IV:] Comunicação com servidores;
	         \begin{enumerate}[itemsep=0em, topsep=0em]
				\item O modelo cliente-servidor;
				\item API's nativas para WEB.
            \end{enumerate}
            
		\item[UNIDADE V:] Persistência de dados; 
	         \begin{enumerate}[itemsep=0em, topsep=0em]
				\item Armazenamento de dados no dispositivo;
                \item Aplicações e Banco de Dados mais comuns para dispositivos móveis;
                \item Relacionando formulários com Banco de Dados;
                \item Visualização de dados.
            \end{enumerate}
            
        \item[UNIDADE VI:] Geolocalização;
	         \begin{enumerate}[itemsep=0em, topsep=0em]
				\item Sistemas de coordenadas geográficas;
         		\item Provedores de localização mais comuns;
                \item Eventos de localização;
				\item Mapas.
            \end{enumerate}

		\item[UNIDADE VII:] Desenvolvimento prático de um sistema para Dispositivos Móveis; 
	         \begin{enumerate}[itemsep=0em, topsep=0em]
				\item  Levantamento de requisitos de software de um aplicativo para dispositivos móveis;
	            \item  Projeto de interface de um aplicativo móvel;
	            \item  Projeto de arquitetura de software para um aplicativo móvel;
	            \item  Projeto de persistência de dados e comunicação para um aplicativo para dispositivos móveis.
            \end{enumerate}
            
        \item[UNIDADE VIII:] Aplicação de teste de software em desenvolvimento de aplicativos.
	         \begin{enumerate}[itemsep=0em, topsep=0em]
				\item  Desenvolvimento de software dirigido por testes;
				\item Técnicas de testes;
				\item Planejamento e execução de testes aplicativo para dispositivos móveis.icas desenvolvidas e construídas a partir da relação ensino-aprendizagem no curso com questões e âmbitos sociais locais e regionais;
                
            \end{enumerate}                                    

	\end{description}
	
	\metodologia
	A disciplina contará com aulas expositivas dialogadas, atividades práticas em laboratório e a realização de trabalhos em equipe e individuais, além da análise e discussão de estudos de caso e a aplicação de metodologias ativas para promover a construção do conhecimento no estudante. A partir das atividades de desenvolvimento de aplicativos realizadas durante a disciplina, fazer uso dos conceitos de testes para criação dos casos de testes e execução destes sobre os aplicativos ou o sistema prático desenvolvidos na disciplina.
	
	\avaliacao
	A avaliação é realizada de forma processual e cumulativa utilizando os instrumentos de avaliação especificados pelo Regulamento de Organização Didática em seu art. 94 \S~1$^\circ$, conforme for mais adequado. A frequência é obrigatória, respeitando os limites de ausência previstos em lei.
	
    
	\clearpage
	\begin{bibbasica}
			
        \item LECHETA, R. R. \textbf{Google Android}: aprenda a criar aplicações para dispositivos móveis com o Android SDK.  ed. rev. e ampl. São Paulo: Novatec, 2013. 824 p. ISBN 9788575223444. 
 		\item LEE, V.; SCHENEIDER, H.; SCHELL, R. \textbf{Aplicações móveis}: arquitetura, projeto e desenvolvimento. Editora Pearson. Livro. 350 p. ISBN 9788534615402. (BVU)
		\item SILVA, Diego (Org.). \textbf{Desenvolvimento para dispositivos móveis}.  São Paulo: Pearson Education do Brasil, 2016.  Livro. 123 p. ISBN 9788543020259. (BVU)
		
	\end{bibbasica}
	
	\begin{bibcomplementar}
    
    	\item  BONATTI, Denilson. \textbf{Desenvolvimento de Jogos em HTML5}. Rio de Janeiro: Editora Brasport, 2014. Livro. 256 p. ISBN 9788574527017. (BVU)
 		\item DEITEL, Paul; DEITEL, Harvey M. \textbf{Java}: como programar. Tradução de Edson Furmankiewicz. Revisão técnica de Fábio Luis Picelli Lucchini. 8. ed. São Paulo: Pearson Prentice Hall, 2010. 1144 p. ISBN 9788576055631. (BVU) 
		\item FLATSCHART, Fábio. \textbf{HTML5}: Embarque imediato. Rio de Janeiro: Brasport, 2011.  256 p. ISBN 9788574525778. (BVU)
		\item LECHETA, Ricardo R. \textbf{Google Android para tablets}: aprenda a desenvolver aplicações para o Android - de smartphones a tablets. São Paulo: Novatec, 2012. 448 p. Inclui bibliografia. ISBN 9788575222928.
		
		\item TANENBAUM, Andrew S.; Bos, Herbert. \textbf{Sistemas operacionais modernos}. Editora Pearson. Livro. 778 p. ISBN 9788543005676. (BVU)
		
	\end{bibcomplementar}
	
	
    	
	
\end{pud}


%%%% Análise e Projeto de Sistemas

\begin{pud}
	
	\pudinfo{TSII.302}{Análise e Projeto de Sistemas}{TSII.209, TSII.210}{20}{20}{2}{3$^\circ$ Semestre}
	
	\ementa
	Fundamentos de Análise e Projeto de Sistemas de Informação. Modelagem de Sistemas. Técnicas de Modelagem: Estruturada e Orientada a Objetos. Linguagem de Modelagem Unificada -- UML. Aplicação de ferramentas computacionais de apoio ao processo de análise e projeto de sistemas. Padrões de projeto. Conceitos de engenharia de software aplicáveis a sistemas. Desenvolvimento da análise e projeto baseado em objetos de um sistema.
	
	\objetivos
	\textbf{OBJETIVO GERAL}:
	Conhecer o processo de Análise e Projeto de Sistemas, aplicando conceitos de engenharia para construção de softwares.	
	\newline\\	
	\textbf{OBJETIVOS ESPECÍFICOS}:
	\begin{itemize}
		
		\item Assimilar as etapas e fundamentos que compõem a análise de sistemas em sistemas computacionais;
		\item Compreender a utilização da UML como uma linguagem de modelagem conceitual e unificada;
		\item Desenvolver diagramas UML para as fases de análise, projeto e implementação de um software orientado a objetos;
		\item Compreender as ferramentas utilizadas para análise de projeto de sistemas. 

	\end{itemize}
	
	\programa
	\begin{description}[itemsep=0em]
		\item[UNIDADE I:] Fundamentos de Análise e Projeto de Sistemas de Informação; 
	         \begin{enumerate}[itemsep=0em, topsep=0em]
				\item Aspectos introdutórios da análise e projeto orientados a objeto;
               	\item Fases da engenharia de requisitos;
               	\item Casos de uso;
            \end{enumerate}
            
        \item[UNIDADE II:] Modelagem;
	         \begin{enumerate}[itemsep=0em, topsep=0em]
				\item Modelagem de aspectos estáticos e dinâmicos de software;
                \item  Linguagem de Modelagem Unificada -- UML;
				\item  Utilização de ferramentas para modelagem UML;
            \end{enumerate}

		\item[UNIDADE III:] Projeto e implementação de sistemas; 
	         \begin{enumerate}[itemsep=0em, topsep=0em]
				\item Principais diagramas da UML (estruturais, comportamentais  e interativos);
				\item Codificação de diagramas;
				\item Padrões de Projeto.
            \end{enumerate}
            
        \item[UNIDADE IV:] Análise e projeto de sistemas para WEB;
	         \begin{enumerate}[itemsep=0em, topsep=0em]
				\item Análise e projeto de sistemas no âmbito WEB;
				\item Especificidades do ambiente WEB.
            \end{enumerate}

                                    

	\end{description}
	
	
	\metodologia
	A disciplina contará com aulas expositivas dialogadas, atividades práticas em laboratório e a realização de trabalhos em equipe e individuais, além da análise e discussão de estudos de caso e a aplicação de metodologias ativas para promover a construção do conhecimento no estudante. As aulas práticas contarão com exemplos implementados em um contexto de linguagem de programação.
	
	\avaliacao
	A avaliação é realizada de forma processual e cumulativa utilizando os instrumentos de avaliação especificados pelo Regulamento de Organização Didática em seu art. 94 \S~1$^\circ$, conforme for mais adequado. A frequência é obrigatória, respeitando os limites de ausência previstos em lei.
	
	
	
	\begin{bibbasica}
		 %\item BEZERRA, Eduardo. \textbf{Princípios de Análise e Projeto de Sistemas com UML}: um guia prático para modelagem de sistemas. 3. ed. Elsevier, 2015. ISBN: 9788535226263
		 \item LARMAN, Craig. \textbf{Utilizando UML e padrões}: uma introdução à análise e ao projeto orientados a objetos e ao desenvolvimento iterativo. Tradução de Rosana T. Vaccare Braga. 3, ed. Porto Alegre: Bookman, 2007. 695 p. Inclui bibliografia. ISBN 9788560031528.
		 %\item WAZLAWICK, R. \textbf{Análise e Projeto de Sistemas de Informação Orientados}. 2. ed. Elsevier Brasil, 2010.
		\item MCLAUGHLIN, Brett; POLLICE, Gary; WEST, David. \textbf{Use a cabeça}: análise e projeto orientado ao objeto. Tradução de Betina Macêdo. Revisão técnica de Eduardo Velasco. Rio de Janeiro: Alta Books, 2007. 442 p. ISBN 9788576081456.
        
		\item MEDEIROS, Ernani Sales de. \textbf{Desenvolvendo software com UML 2.0}: definitivo. São Paulo:  Pearson Makron Books, 2004. Livro. 288 p. ISBN 9788534615297. (BVU)
	\end{bibbasica}
	
	\begin{bibcomplementar}
    
 		\item DEITEL, Paul; DEITEL, Harvey. \textbf{Java}: como programar 8. ed. São Paulo: Pearson Education do Brasil, 2017. ISBN 9788576055631. (BVU)
		\item FOWLER, Martin. \textbf{UML essencial}: um breve guia para a linguagem - padrão de modelagem de objetos. Tradução de João Eduardo Nóbrega Tortello. 3. ed. Porto Alegre: Bookman, 2005. 160 p. ISBN 8536304545.
		% \item FILHO, Wilson de Pádua Paula. Engenharia de Software: fundamentos, métodos e padrões. 3. ed. Rio de Janeiro: LTC, 2009.
		%\item MACHADO, F.N.R. \textbf{Análise e Gestão de Requisitos de Software}: onde nascem os sistemas. Editora Saraiva, 2018.
		\item PAGE-JONES, Meilir. \textbf{Fundamentos do Desenho Orientado a Objeto com UML}. Editora Pearson. Livro. 488 p. ISBN 9788534612432. 
		\item PRESSMAN, Roger S.; MAXIM, Bruce R. \textbf{Engenharia de software}: uma abordagem profissional. Tradução de Ariovaldo Griesi, Mario Moro Fecchio. Revisão técnica de Reginaldo Arakaki, Renato Manzan de Andrade, Julio Arakaki. 7. ed. Porto Alegre: AMGH, 2011. 780 p. Inclui bibliografia. ISBN 9788563308337.				 
		
		\item SOMMERVILLE, Ian. \textbf{Engenharia de software}. Tradução de Kalinka Oliveira, Ivan Bosnic. Revisão técnica de Kechi Hirama. 9. ed. São Paulo: Pearson Prentice Hall, 2011. 529 p. Inclui bibliografia. ISBN 9788579361081.	
		
		
	\end{bibcomplementar}
	
	
    	
	
\end{pud}


%%% DevOps

\begin{pud}
	\pudinfo{TSII.303}{Desenvolvimento e Operações}{TSII.102 e TSII.105}{20}{20}{2}{3$^\circ$ Semestre}

	\ementa
	Revisão dos conceitos de sistemas operacionais, redes de computadores e
servidores aplicados ao desenvolvimento e operações. Manipulação de containers.
Integração e entrega contínua. Monitoramento, avaliação de desempenho e
processos de implementação.

	\objetivos
	Melhorar a qualidade do software, automatizar e monitorar todas
as operações, realizando testes, integrações e entregas contínuas.
	\begin{itemize}
	  \item Integrar os conceitos de servidores, sistemas operacionais e redes de
computadores ao dia a dia do desenvolvedor de operações;
\item Conhecer e especializar-se com ambientes de desenvolvimento.
	\end{itemize}
	
	\programa
	\begin{description}[itemsep=0em]

		\item[UNIDADE I:] Fundamentos; 
        \begin{enumerate}[itemsep=0em, topsep=0em]
			\item Elementos;
			\item Serviços;
			\item Sistemas Operacionais;
			\item Rede de computadores;
			\item Servidor Web (Apache, Nginx e IIS).
		\end{enumerate}
		
		\item[UNIDADE II:] Infraestrutura como Código; 
        \begin{enumerate}[itemsep=0em, topsep=0em]
			\item Conteinerização;
			\item Plataforma de nuvem;
			\item Provisionamento de infraestrutura.
		\end{enumerate}
		
		\item[UNIDADE III:] Integração e Entrega Contínua; 
        \begin{enumerate}[itemsep=0em, topsep=0em]
			\item Definições;
			\item Rotina de integração contínua;
			\item Pipeline;
			\item Teste e rotinas para comandos.
		\end{enumerate}
		
		\item[UNIDADE IV:] Monitoramento e avaliação de desempenho; 
        \begin{enumerate}[itemsep=0em, topsep=0em]
			\item Infraestrutura de monitoramento;
			\item Aplicações de monitoramento;
			
		\end{enumerate}
		
		\item[UNIDADE V:] Processos de implementação; 
        \begin{enumerate}[itemsep=0em, topsep=0em]
			\item Análise e levantamento de requisitos;
			\item Planejamento e Implementação.
			
		\end{enumerate}
	\end{description}
	
	\metodologia
	Aulas expositivas e interativas com uso de recursos audiovisuais. Exercícios
práticos de implementação com o uso de softwares específicos.

	\recursos
	Data-show, pincel e quadro branco, laboratório de informática, laboratório de redes
de computadores, computadores, softwares para apoio em classe e extraclasse e
softwares específicos.

	\avaliacao
	A avaliação é realizada de forma sistemática, periódica e cumulativa utilizando os
instrumentos de avaliação especificados pelo Regulamento de Organização
Didática (ROD) em seu art. 94 § 1º, conforme for mais adequado. A frequência é
obrigatória, respeitando os limites de ausência previstos em lei.
	\naopresencial
	
	\begin{bibbasica}
		\item FREEMAN, Emily. \textbf{DevOps Para Leigos}. Rio de Janeiro: Editora Alta Books,
2021. E-book. ISBN 9788550816661. (MB)
		\item JERÔNIMO, Anderson Pereira de Lima. \textbf{Práticas da cultura DevOps no
desenvolvimento de sistemas}. São Paulo: Platos Soluções Educacionais S.A.,
2021. E-book. ISBN 9786553560567. (MB)
		\item MONTEIRO, E. R.; CERQUEIRA, Marcos V. Bião; SERPA, Matheus da Silva et al.
\textbf{DevOps}. Porto Alegre: SAGAH, 2021. E-book. ISBN 9786556901725. (MB)
	\end{bibbasica}
	
	
	\begin{bibcomplementar}
		\item BORGES, Fábio Roberto. \textbf{Transformação Digital}: Um Guia Prático Para Liderar
Empresas que se Reinventam. Rio de Janeiro: Atlas, 2021. E-book. ISBN
9788597027433. (MB)
		\item MANZANO, José Augusto Navarro Garcia. \textbf{Estudo Dirigido de Linguagem C}.
São Paulo: Érica, 2002. E-book. ISBN 9788536519128. (MB)
SILVA, Fernanda Rosa; SOARES, Juliane Adélia; SERPA, Matheus da S. Cloud
Computing. Porto Alegre: SAGAH, 2020. E-book. ISBN 9786556900193. (MB)
		\item TANENBAUM, Andrew S.; WOODHULL, Albert S. \textbf{Sistemas operacionais}:
projeto e implementação. 3.ed. Porto Alegre: Bookman, 2008. 653 p. ISBN
9788577800575.
		\item WANDERLEY, Alex R. M. C.; PONTUAL, Ricardo de Almeida. \textbf{Gerenciamento de
Servidores}. São Paulo: Érica, 2019. E-book. ISBN 9788536532103. (MB)
	\end{bibcomplementar}
\end{pud}



%%% PIM

\begin{pud}
	\pudinfo{TSII.304}{Projeto Integrador Multidisciplinar}{TSII.206}{20}{20}{2}{3$^\circ$ Semestre}
	
	\ementa
	Controle e monitoramento do projeto. Desenvolvimento da proposta de projeto. Validação e implantação da solução. Apresentação dos resultados obtidos.

	\objetivos
	Implementar uma solução de software para problemas do mundo real, integrando conhecimentos multidisciplinares.
	
	\begin{itemize}
	  \item Promover a integração multidisciplinar entre as disciplinas de Engenharia de Software, Banco de Dados, Programação Orientada a Objetos, Programação WEB I e II, Programação para Dispositivos Móveis e Gestão de Projetos;
		\item Pensar a pesquisa e o desenvolvimento de um projeto a partir de uma perspectiva multidisciplinar;
		\item Compreender as etapas de implementação, teste e entrega de sistemas;
		\item Permitir a experiência com implementação de sistemas voltados a problemas reais;
		\item Compreender a importância de ações de extensão para o fortalecimento do relacionamento entre a instituição e a sociedade.
	\end{itemize}


	\programa
	\begin{description}[itemsep=0em]
 		\item[UNIDADE I:] Revisão do Projeto; 
 	    \begin{enumerate}[itemsep=0em, topsep=0em]
 			\item Definição das equipes e projetos;
 			\item Revisão do escopo e dos requisitos do projeto;
 			\item Negociação e priorização dos requisitos;
 			\item Definição do cronograma de desenvolvimento.
		\end{enumerate}
		
		\item[UNIDADE II:] Etapa de Implementação do Projeto; 
 	    \begin{enumerate}[itemsep=0em, topsep=0em]
 			\item Codificação dos módulos e unidades da solução proposta;
 			\item Testes unitários e de integração;
 			\item Implementação da base de dados e integração com a aplicação;
 			\item Controle e monitoramento do desenvolvimento do projeto;
 			\item Entrega da primeira versão funcional e apresentação em sala de aula.
		\end{enumerate}
		
		\item[UNIDADE III:] Etapa de Testes e Validação da Proposta; 
 	    \begin{enumerate}[itemsep=0em, topsep=0em]
 			\item Testes de sistema junto aos stakeholders;
 			\item Documentação de feedbacks dos usuários e possíveis ajustes;
 			\item Implementação de alterações ou correções de erros;
 			\item Controle e monitoramento do desenvolvimento do projeto;
 			\item Apresentação dos feedbacks e alterações em sala de aula.
		\end{enumerate}
		
		\item[UNIDADE IV:] Entrega e Encerramento do Projeto. 
 	    \begin{enumerate}[itemsep=0em, topsep=0em]
 			\item Implantação da versão final da solução;
 			\item Reunião de avaliação e encerramento do projeto;
 			\item Desenvolvimento de um artigo científico, relatório técnico ou peça
equivalente sobre a solução desenvolvida.
		\end{enumerate}
		
		
	\end{description}
	
	\metodologia
	Aulas expositivas e interativas com uso de recursos audiovisuais. O professor deve
	conduzir as principais etapas para o desenvolvimento de uma solução de software
	(web e/ou mobile) que atenda às necessidades de alguma entidade externa à
	instituição. Nesse caso, deve atuar orientando a sequência de atividades que devem
	ser realizadas pelos alunos, administrando o tempo, garantindo o cumprimento de
	metas e avaliando a produção feita por esses. Os projetos a serem desenvolvidos
	são especificados na disciplina de Projeto Integrador Multidisciplinar I.
	
	A definição das equipes será feita pelos alunos sob orientação do professor, que
	pode intervir nas escolhas caso necessário para a adequada condução do projeto.
	Os projetos a serem desenvolvidos são especificados na disciplina de Projeto
	Integrador Multidisciplinar I e são baseados em necessidades reais específicas de
	estabelecimentos comerciais, instituições de ensino, setores empresariais ou
	organizações sociais da região. O professor deve então deixar os alunos cientes do
	caráter extensionista da proposta a ser desenvolvida, mostrando a importância
	dessa ação para o fortalecimento do relacionamento entre a instituição e a
	sociedade.
	
	As equipes definirão junto ao professor o modelo de processo de software que irão
	seguir. Assim, pode-se optar por um modelo mais clássico (sequencial linear) ou por
	um método ágil de desenvolvimento (iterativo e incremental). Ao final de cada etapa,
	ou cada iteração (ou conjunto de iterações), o professor pode solicitar, além da
	documentação atualizada do projeto, que as equipes apresentem suas produções
	em formato de seminário. No final do semestre letivo, o professor pode organizar
	um momento para a apresentação final das propostas, convidando os stakeholders
	demandantes dos projetos a se fazerem presentes na instituição.
		
	\recursos
	Data-show, pincel e quadro branco, laboratório de informática, computadores, softwares para apoio em classe e extraclasse e aplicativos específicos da área.

	\avaliacao
	A avaliação é realizada de forma sistemática, periódica e cumulativa utilizando os
instrumentos de avaliação especificados pelo Regulamento de Organização
Didática (ROD) em seu art. 94 § 1º, conforme for mais adequado. A frequência é
obrigatória, respeitando os limites de ausência previstos em lei.
	\naopresencial

	Além disso, sugere-se a definição de um cronograma de entregas junto às equipes, que deve
ser cumprido sob pena de redução da nota em casos de atrasos. A entrega pode
ser composta pela documentação atualizada do projeto, sendo avaliados critérios
como organização, clareza das informações, correta utilização das técnicas
propostas e cumprimento das metas estabelecidas. Ainda, o professor pode solicitar
uma apresentação em formato de seminário, avaliando critérios como utilização do
tempo, clareza, objetividade, capacidade de argumentação, qualidade do material
exposto e cumprimento das metas estabelecidas.
	
	\begin{bibbasica}
		\item ALVES, William P. \textbf{Projeto de sistemas web}: conceitos, estruturas, criação de banco de dados e ferramentas de desenvolvimento. São Paulo: Érica, 2015. ISBN 9788536532462.
		\item ELMASRI, Ramez. E.; NAVATHE, Shamkant B. \textbf{Sistemas de banco de dados}. 7. ed. São Paulo: Pearson Education do Brasil, 2019. 1126 p. ISBN 9788543025001.
		\item LECHETA, R. R. \textbf{Google Android}: aprenda a criar aplicações para dispositivos móveis com o Android SDK. 3. ed. São Paulo: Novatec, 2013. ISBN 9788575222447.
	\end{bibbasica}
	
	\begin{bibcomplementar}
		\item BARBOSA, S. D. J.; SILVA, B. S. \textbf{Interação Humano-Computador}. Rio de
Janeiro: Campus-Elsevier, 2010.
		\item BARNES, David J.; KÖLLING, Michael. \textbf{Programação orientada a objetos com
Java}: uma introdução prática usando o BLUEJ. 4. ed. São Paulo: Pearson
Prentice Hall, 2009. 455 p. ISBN 9788576051879.
		\item CLEMENTS, James P.; GIDO, Jack. \textbf{Gestão de projetos}. 3ª reimpr. da 2. ed. São
Paulo: Cengage Learning, 2016. 511 p. ISBN: 9788522112760.
		\item PRESSMAN, Roger S.; MAXIM, Bruce R. \textbf{Engenharia de Software}: uma abordagem profissional. 9. ed. Porto Alegre: AMGH, 2021. 672 p., il, 28 cm. ISBN 9786558040101.
		\item PRIKLADNICKI, R.; WILLI, R.; MILANI, F. \textbf{Métodos ágeis para desenvolvimento de Software}. Porto Alegre: Bookman, 2014. ISBN: 9788582602072. (MB)
	\end{bibcomplementar}
\end{pud}


%%% Desenvolvimento WEB II

\begin{pud}
	
	\pudinfo{TSII.305}{Desenvolvimento WEB II}{TSII.201 e TSII.206}{40}{40}{4}{3$^\circ$ Semestre}
	
	\ementa
	Linguagem de desenvolvimento \textit{back-end}: Orientação a Objetos. \textit{Frameworks} WEB: Introdução à \textit{frameworks} para desenvolvimento WEB. Acesso a banco de dados em sistemas WEB via \textit{framework} de Mapeamento objeto-relacional (ou ORM, do inglês: \textit{Object-relational mapping}). Engenharia WEB: Desempenho com \textit{cluster}, balanceamento de carga, alta disponibilidade, criptografia, SQL \textit{injection}.
	
	\objetivos
	Construir sistemas Web observando as boas práticas de construção com \textit{Frameworks} de desenvolvimento WEB Orientação a Objetos com ORM ou tecnologias similares.  Analisar desempenho de sistemas WEB.
	
	\metodologia
	Aulas expositivas e interativas com uso de recursos audiovisuais. Atividades práticas relacionadas a técnicas e métodos atuais de Desenvolvimento WEB com \textit{Frameworks}. Além disso, trabalhos individuais e em grupo, utilização de laboratório com exercícios práticos que possam auxiliar o treinamento e desenvolvimento de aplicações para WEB com \textit{Frameworks}, envolvendo todos os aspectos aprendidos durante a disciplina e o curso.    
	
	
	\programa
	\begin{description}[itemsep=0em]
		\item[UNIDADE I:] Linguagem de desenvolvimento \textit{back-end} Orientação a Objetos;
	    \item[UNIDADE II:]  \textit{Framework} WEB Orientado a Objetos;
        \item[UNIDADE III:] \textit{Framework} de Mapeamento Orientado a Objetos (ORM);
        \item[UNIDADE IV:] Projeto de Sistemas WEB com \textit{Frameworks}.
	\end{description}
	
	
	\avaliacao
	A avaliação é realizada de forma processual e cumulativa utilizando os instrumentos de avaliação especificados pelo Regulamento de Organização Didática em seu art. 94 \S~1$^\circ$, conforme for mais adequado. A frequência é obrigatória, respeitando os limites de ausência previstos em lei.
	
	
	\clearpage
	
	\begin{bibbasica}
			
        \item DALL'OGLIO, Pablo. \textbf{PHP}: programando com orientação a objetos. 2. ed. São Paulo: Novatec Editora, 2009.  ISBN 9788575222003.
		\item DEITEL, Paul J.; DEITEL, Harvey M.  \textbf{Ajax, Rich Internet Applications e Desenvolvimento Web para Programadores}. Editora Pearson. 776 p. ISBN 9788576051619. (BVU)
		%\item OLIVIERO, Carlos A. J. \textbf{Faça um site PHP 5.2 com MySQL 5.0}: comércio eletrônico orientado por projeto. São Paulo: Érica, 2010.
		\item SOARES, Walace. \textbf{Crie um framework para sistemas web e com PHP 5 e Ajax}. São Paulo: Érica, 2009.  ISBN 9788536502373.
		
	\end{bibbasica}
	
	\begin{bibcomplementar}
    
	    
		\item FLATSCHART, F\'abio. \textbf{HTML 5}: embarque Imediato. Rio de Janeiro: Editora Brasport, 2011. 256 p. ISBN 9788574525778. (BVU)
		\item LEMAY, Laura; COLBURN, Rafe; TYLER, Denise. \textbf{Aprenda a Criar P\'aginas Web com HTML e XHTML em 21 Dias}.  São Paulo: Pearson Education do Brasil, 2002.  1158 p. ISBN 9788534614283. (BVU)
 		
 		\item MARINHO, Antônio Lopes (Org.). \textbf{Desenvolvimento de aplicações para Internet}. São Paulo: Pearson Education do Brasil, 2016. Livro. 139 p. ISBN 9788543020112. (BVU)
 		\item PUGA, Sandra; FRANCA, Edson; GOYA, Milton. \textbf{Banco de dados}: Implementa\c{c}\~ao em SQL, PL/SQL e Oracle 11g.  São Paulo: Pearson Education do Brasil, 2013. 356 p.  ISBN 9788581435329. (BVU)
 		\item SEGURADO, Valquiria Santos (Org.). \textbf{Projeto de interface com o usuário}. São Paulo: Pearson Education do Brasil, 2015. Livro. 195 p. ISBN 9788543017303. (BVU) 
 		%\item SILVA, Maur\'icio Samy. \textbf{JavaScript}: guia do programador. S\~ao Paulo: Novatec, 2010. 604 p. Inclui bibliografia e \'indice. ISBN 9788575222485.  
		 
	\end{bibcomplementar}
	
	
	
\end{pud}
% 
% %%%%% SEGURANCA DE REDES
% \begin{pud}
% 	
% 	\pudinfo{TSII.317}{Segurança de Redes}{TSII.212}{20}{20}{2}{3$^\circ$ Semestre}
% 	
% 	\ementa
% 	Fundamentos de gerenciamento em redes de computadores. Protocolos de gerenciamento de redes, Metodologias para resolução de problemas em redes. Ferramentas para gerenciamento. Fundamentos de segurança da informação, Introdução à criptografia. \textit{Firewalls}. Detecção e prevenção de invasões. Segurança em redes sem fio. 
% 	
% 	
% 	\objetivos
% 	Fornecer ao aluno conhecimento para garantir funcionamento, manutenção e reparos em redes de computadores. Abordar aspectos de gerenciamento e políticas de segurança da informação.
% 	
% 	\programa
% 	\begin{description}[itemsep=0em]
% 		\item[UNIDADE I:]Fundamentos de Segurança da Informação; 
% 	         \begin{enumerate}[itemsep=0em, topsep=0em]
% 				\item Definições de segurança;
% 				\item A segurança da informação;
% 				\item Modelos de referência da segurança;
% 				\item Plano de segurança;
% 				\item Análise e gerenciamento de riscos;
% 				\item Política de segurança.
% 	        \end{enumerate}
% 	        
% 	    \item[UNIDADE II:] Criptografia;
% 	         \begin{enumerate}[itemsep=0em, topsep=0em]
% 				\item Histórico;
% 				\item Terminologia;
% 				\item Usos da criptografia;
% 				\item Chaves criptográficas;
% 				\item Algoritmos de criptografia;
% 				\item Tipos de criptografia;
% 				\item Funções \textit{hash};
% 				\item Certificação digital.
% 	        \end{enumerate}
% 	
% 		\item[UNIDADE III:] \textit{Firewalls};
% 	        
% 	    \item[UNIDADE IV:] Detecção e Prevenção de Invasões;
% 	
% 	        
% 	    \item[UNIDADE II:] Tipos de Ataques e Contramedidas.
% 	         \begin{enumerate}[itemsep=0em, topsep=0em]
% 				\item Vírus;
% 				\item \textit{Phishing};
% 				\item Negação de serviço (DoS e DDoS);
% 				\item \textit{Spoofing};
% 				\item \textit{Aircrack}.
% 	        \end{enumerate}
% 	                                
% 	
% 	\end{description}
% 	
% 	\metodologia
% 	Aulas teóricas sobre gerência e segurança de redes e aulas práticas utilizando ferramentas de gerenciamento de rede. Além disso, ocorrerão aulas práticas para execução de um projeto de segurança em redes de computadores.
% 
% 
% 	\avaliacao
% 	A avaliação é realizada de forma processual e cumulativa utilizando os instrumentos de avaliação especificados pelo Regulamento de Organização Didática em seu art. 94 \S~1$^\circ$, conforme for mais adequado. A frequência é obrigatória, respeitando os limites de ausência previstos em lei.
% 	
% 		
% 	
% 	\begin{bibbasica}
% 	
% 		\item BOMFATI, Cláudio Adriano; KOLBE JUNIOR, Armando. \textbf{Crimes cibernéticos}. Editora Intersaberes. Livro. 198 p. ISBN 9788522702916. (BVU)
% 		\item KUROSE, James F. \textit{et al}. \textbf{Redes de computadores e a internet}: uma abordagem top-down. 6. ed. São Paulo, SP: Pearson Education do Brasil, 2013. 634 p. Bibliografia. ISBN 9788581436777. (BVU)
% 		\item STALLINGS, William. \textbf{Criptografia e segurança de redes}: princípios e práticas. 4. ed.  Editora Pearson. Livro. 580 p. ISBN 9788543005898. (BVU)
% 		
% 		
% 		
% 	\end{bibbasica}
% 	
% 	\begin{bibcomplementar}
% 	
% 		\item ALENCAR, Marcelo Sampaio de. \textbf{Engenharia de redes de computadores}. São Paulo: Érica, 2012. 286 p. Bibliografia. ISBN 9788536504117.
% 		\item COULOURIS, George; DOLLIMORE, Jean; KINDBERG, Tim. \textbf{Sistemas distribuídos}: conceitos e projeto. Tradução de João Eduardo Nóbrega Tortello. Revisão técnica de Alexandre da Silva Carissimi. 4. ed. Porto Alegre: Bookman, 2007. 784 p. Inclui bibliografia. ISBN 9788560031498.
% 		\item SOARES, Walace; FERNANDES, Gabriel. \textbf{Linux}: fundamentos. São Paulo: Érica, 2010. 206 p. Inclui referência e índice. ISBN 9788536503219.
% 		%\item SOUSA, Lindeberg Barros de. \textbf{Projetos e implementação de redes}: fundamentos, soluções, arquiteturas e planejamento. 3. ed. , rev. São Paulo: Érica, 2013. 318 p. ISBN 9788536501666.		
% 		\item TANENBAUM, Andrew S. ; WETHERALL, David J. \textbf{Redes de computadores}. Tradução de Daniel Vieira. Revisão técnica de Benedito Isaías Lima Lopes. 5. ed. São Paulo: Pearson Prentice Hall, 2011. 582 p. Inclui bibliografia 9788576059240. (BVU)
% 		\item TANENBAUM, Andrew S.; STEEN, Maarten Van. \textbf{Sistemas distribuídos}: princípios e paradigmas. Tradução de Arlete Simille Marques. Revisão técnica de Wagner Luiz Zucchi. 2. ed. São Paulo: Pearson Prentice Hall, 2007. 402 p. Inclui bibliografia. ISBN 9788576051428. (BVU)
% 		
% 	\end{bibcomplementar}
% 	
% 	
% 	
% \end{pud}



%%%% GESTAO DE PROJETOS

\begin{pud}

	\pudinfo{TSII.306}{Gestão de Projetos}{TSII.204}{36}{4}{2}{3$^\circ$ Semestre}
	
	\ementa
	Introdução ao gerenciamento de projetos. Ciclo de vida de um projeto. Análise de riscos e custos. Gerenciamento da qualidade. Liderança e trabalho em equipe. Avaliação de resultados de um projeto. Melhores práticas em gerenciamento de projetos.
	
	\objetivos
	Conhecer os conceitos e práticas da gerência de projetos. Aprender as técnicas e ferramentas de gerenciamento de projetos na concepção, planejamento, implementação, controle e conclusão de atividades de projeto de software.  Conhecer as práticas e ferramentas de gerenciamento de projetos.
	
	
	\programa
	\begin{description}[itemsep=0em]
		\item[UNIDADE I:] Introdução ao gerenciamento de projetos;
		\begin{enumerate}[itemsep=0em, topsep=0em]
			\item Conceitos e tipos de projetos;
			\item Portifólio, programa e projeto;
			\item Origem e evolução do gerenciamento de projetos;
			\item Ciclo de vida de um projeto;
			\item Escopo, tempo e dinheiro de um projeto;
			\item Metodologias de gerenciamento de projetos;
			\item Ferramentas no gerenciamento de projetos.
		\end{enumerate}
		
	    \item[UNIDADE II:] Ciclo de vida de um projeto;
		\begin{enumerate}[itemsep=0em, topsep=0em]
			\item Processos de iniciação, execução, controle e encerramento de projetos;
			\item Estrutura analítica do projeto;
			\item Cronograma e métodos de avaliação e acompanhamento;
			\item Alocação de recursos no projeto.
		\end{enumerate}
		
        \item[UNIDADE III:] Análise de riscos e custos;
		\begin{enumerate}[itemsep=0em, topsep=0em]
			\item  Definição de risco e custos;
			\item  Identificação e categorização de riscos;
			\item  Estratégias e plano de resposta aos riscos.
		\end{enumerate}

        \item[UNIDADE IV:] Gerenciamento da qualidade;
		\begin{enumerate}[itemsep=0em, topsep=0em]
			\item Definição de qualidade;
			\item Qualidade de produto/processo/projeto;
			\item Planejamento da qualidade;
			\item Processos de auditorias e inspeções;
			\item Certificação.
		\end{enumerate}
		
		\item[UNIDADE V:] Liderança e trabalho em equipe;
		\begin{enumerate}[itemsep=0em, topsep=0em]
			\item  Liderança versus gerência;
			\item  Características dos líderes e estilos de liderança;
			\item  Vantagens de trabalho em equipe;
			\item  Liderança compartilhada;
			\item  Flexibilidade e adaptabilidade.
		\end{enumerate}
		
		\item[UNIDADE VI:]  Avaliação de resultados de um projeto;
				
		\item[UNIDADE VII:]  Melhores práticas em gerenciamento de projetos.
		
	\end{description}
	
	\metodologia
	Aulas teóricas: expositivo-dialogadas com aplicação e resolução de exercícios, estudos dirigidos, seminários, vídeos e dinâmicas de grupo. Aulas práticas: realizadas em jogos simulados, laboratório, visitas técnicas e/ou participações em eventos. Recursos: quadro branco, cartolina, pincéis, post-it, lousa digital, data-show, aparelho de som, computador pessoal, smartphone, Internet, e-mail; redes sociais, Ambiente Virtual de Aprendizagem (AVA) e outros.
	
	
	\avaliacao
	Verificação de conhecimentos através de avaliação presencial, avaliação a distância desenvolvidas em Ambiente Virtual de Aprendizagem empregando a metodologia de avaliação disponível no Google Sala de Aula e auto avaliação permitindo ao aluno saber seu desempenho. A avaliação será desenvolvida nas seguintes formas:
	\begin{description}[itemsep=0em, ]
		\item[$\bullet$ Diagnóstica --] levantamento dos conhecimentos prévio dos alunos;
        \item[$\bullet$ Continuada --] análise de todo o processo de ensino-aprendizagem observando a participação individual e em grupo, o envolvimento nas atividades, o desenvolvimento dos conteúdos e o nível de percepção apresentado, isto é, o olhar não apressado que consegue descobrir detalhes, estabelecer comparações e conexões com o dia-a-dia, a condição humana, enfim, a própria vida;
	\end{description}
        
    Tipos de verificação do conhecimento construído durante as aula: 
	\begin{description}[itemsep=0em, ]
		\item[$\bullet$ Escrita,] através de questionário individual e/ou equipe;
        \item[$\bullet$ Oral,] através de apresentação individual e/ou equipe;
	\end{description}
	Os recursos avaliativos serão utilizados com base  no art. 94 \S~1$^\circ$ alínea de I a XV do Regulamento de Organização Didática.


	\clearpage
	\begin{bibbasica}
		
		\item CLEMENTS, James P.; GIDO, Jack. \textbf{Gestão de projetos}. 3\textordfeminine\ reimpr. da 2. ed. São Paulo: Cengage Learning, 2016. 511 p. ISBN 978-85-221-1276-0.	
        \item KERZNER, Harold; RIBEIRO, Lene Belon; BORBA, Gustavo Severo de. \textbf{Gestão de projetos}: as melhores práticas. 2. ed. Porto Alegre: Bookman, 2006. 824 p. Inclui Bibliografia. ISBN 9788536306186.
		\item VALERIANO, Dalton L. \textbf{Gerenciamento Estratégico e Administração por Projetos}. Editora Pearson. Livro. 324 p. ISBN 9788534612081. (BVU) 	
		
	\end{bibbasica}
	
	\begin{bibcomplementar}
    	
    	\item PROJECT MANAGEMENT INSTITUTE. \textbf{Um Guia do conhecimento em gerenciamento de Projetos}:  guia PMBOK\textregistered. 5. ed. São Paulo: Saraiva, 2014.  589 p. ISBN 9788502223721.
		\item SABBAG, Paulo Yazigi. \textbf{Gerenciamento de projetos e empreendedorismo}. 2. ed. São Paulo: Saraiva, 2013. 226 p. Inclui bibliografia. ISBN 9788502204447.
		\item VALLE, André Bittencourt do. \textbf{Fundamentos do gerenciamento de projetos}. 2. ed. Rio de Janeiro: FGV, 2010. 172 p. (Gerenciamento de projetos). ISBN 9788522507986.
		\item VALERIANO, Dalton. \textbf{Moderno gerenciamento de projetos}. 2.ed Editora Pearson. Livro. 284 p. ISBN 9788543004518. (BVU)
		\item XAVIER, Carlos Magno da Silva. \textbf{Gerenciamento de projetos}: como definir e controlar o escopo do projeto. 2. ed. São Paulo: Saraiva, 2009. 258 p.  ISBN 9788502061958. 	
		
	\end{bibcomplementar}
	
\end{pud}


%%%% TESTES

\begin{pud}

	\pudinfo{TSII.307}{Testes e Qualidade de Software}{TSII.202}{20}{20}{2}{3$^\circ$ Semestre}
	
	\ementa
	Fundamentos da qualidade de software. Modelos de referência para qualidade de
software. Métricas. Fundamentos de Teste de Software. Testes Automatizados e
Testes Ágeis.

	\objetivos
	Aplicar técnicas para manter e avaliar a qualidade de sistemas e processos de desenvolvimento de software.
	\begin{itemize}
	  \item Compreender os fundamentos de qualidade de software;
	  \item Conhecer os modelos de referências mais utilizados;
	  \item Dominar as métricas de qualidade de software;
	  \item Desenvolver diferentes tipos de testes de software.
	\end{itemize}
	
	\programa
	\begin{description}[itemsep=0em]
		\item[UNIDADE I:] Fundamentos da qualidade de software;
		\begin{enumerate}[itemsep=0em, topsep=0em]
			\item Contextualização do mercado de Tecnologia da Informação;
			\item Contextualização do mercado de Garantia de Qualidade (QA);
			\item Importância da qualidade de software;
			\item Perfis e responsabilidade de um QA;
			\item Qualidade do produto;
			\item Qualidade do processo.
		\end{enumerate}
		
		\item[UNIDADE II:] Modelos de referência para qualidade de software;
		\begin{enumerate}[itemsep=0em, topsep=0em]
			\item Modelo CMMI-DEV;
			\item Modelo MPS.BR-SW.
		\end{enumerate}
		
		\item[UNIDADE III:] Métricas;
		\begin{enumerate}[itemsep=0em, topsep=0em]
			\item Métricas para teste de software;
			\item Criação de métricas e resultados da equipe;
			\item Métrica de processos.
		\end{enumerate}
		
		\item[UNIDADE IV:] Fundamentos de Teste de Software;
		\begin{enumerate}[itemsep=0em, topsep=0em]
			\item Definições e princípios de testes;
			\item Testes durante o ciclo de desenvolvimento de software;
			\item Plano de testes e documentação;
			\item Níveis de teste – unidade, integração, sistema, aceitação, alfa, beta e regressão;
			\item Técnicas de teste: Caixa branca e caixa preta;
			\item Tipos de teste: funcionalidade, desempenho, usabilidade, segurança,
portabilidade e stress.
		\end{enumerate}
		
		\item[UNIDADE V:] Testes automatizados e testes ágeis;
		\begin{enumerate}[itemsep=0em, topsep=0em]
			\item Suíte de testes e casos de testes;
			\item Automação de testes;
			\item Manutenção de testes;
			\item Documentação de testes;
			\item \textit{Test Driven Development} (TDD);
			\item \textit{Behavior Driven Development} (BDD);
			\item Testes de Interface.
			
		\end{enumerate}
		
		\item[UNIDADE VI:] Gerenciamento de Testes.
		\begin{enumerate}[itemsep=0em, topsep=0em]
			\item Organização do teste;
			\item Planejamento e estimativas de teste;
			\item Monitoramento e controle dos testes;
			\item Gerenciamento de configurações;
			\item Riscos e testes;
			\item Gerenciamento de defeitos.
		\end{enumerate}
		
	\end{description}

	\metodologia
	Aulas expositivas e interativas com uso de recursos audiovisuais. Atividades de
pesquisa individuais e em grupo em artigos científicos e repositórios de código-
fonte. Atividades práticas com estudos de caso de testes e de análise de garantia
de qualidade. Atividades de análise de projetos de software para aplicação de
métricas e testes.

	\recursos
	Data-show, pincel e quadro branco, laboratório de informática, computadores,
softwares para apoio em classe e extraclasse, repositórios de código-fonte e
softwares específicos da área de testes e de qualidade.

	\avaliacao
	A avaliação é realizada de forma sistemática, periódica e cumulativa utilizando os
instrumentos de avaliação especificados pelo Regulamento de Organização
Didática (ROD) em seu art. 94 § 1º, conforme for mais adequado. A frequência é
obrigatória, respeitando os limites de ausência previstos em lei.
	\naopresencial
	
	\begin{bibbasica}
		\item KOSCIANSKI, André; SOARES, Michel dos Santos. \textbf{Qualidade de Software}. 2
ed. Novatec, 2006. ISBN 9788575221129.
		\item RIOS, Emerson; MOREIRA FILHO, Trayahú R. \textbf{Teste de software}. 3. ed. Rio de
Janeiro: Alta Books, 2013. 296 p. ISBN 9788576087755.
		\item SOMMERVILLE, Ian. \textbf{Engenharia de software}. 10. ed. São Paulo: Pearson, 2019.
529 p. ISBN 9788543024974.
	\end{bibbasica}
	
	
	\begin{bibcomplementar}
		\item ANICHE, Maurício. \textbf{Testes Automatizados de Software}. Casa do Código, 2015.
166 p. ISBN 9788555190285.
		\item FÉLIX, Rafael. \textbf{Teste de software}. São Paulo: Pearson 2016 139 p. ISBN
9788543020211.
		\item GIOCONDO, Marino Antonio Gallotti. \textbf{Qualidade de software}. São Paulo:
Pearson, 2015. 139 p. ISBN 9788543020358.
		\item GONÇALVEZ, Priscila de F.; BARRETO, Jeanine dos S.; ZENKER, Aline M.
\textbf{Testes de software e gerência de configuração}. Grupo A, 2019. E-book. ISBN
9788595029361. (MB)
		\item ZANIN, Aline; JÚNIOR, Paulo A P.; ROCHA, Breno C. \textbf{Qualidade de software}.
Grupo A, 2018. E-book. ISBN 9788595028401. (MB)
	\end{bibcomplementar}

\end{pud}

%%%% OPTATIVAS



%%%% Artes
% level 1
\renewcommand{\theenumi}{\arabic{enumi}}
\renewcommand{\labelenumi}{\theenumi.}
% level 2
\renewcommand{\theenumii}{\arabic{enumii}}
\renewcommand{\labelenumii}{\theenumi.\theenumii}
% level 3
\renewcommand{\theenumiii}{\arabic{enumiii}}
\renewcommand{\labelenumiii}{\theenumi.\theenumii.\theenumiii}
% level 4
\renewcommand{\theenumiv}{\arabic{enumiv}}
\renewcommand{\labelenumiv}{\theenumi.\theenumii.\theenumiii.\theenumiv}

\begin{pud}

	\pudinfo{TSII.401}{Artes}{--}{20}{20}{2}{Optativa}
	
	\ementa
	Apresentação e discussão sobre aspectos histórico-sociais que envolvem a produção em música em diálogo com a tecnologia. Utilização de ferramentas computacionais para a criação em música.
	
	
	\objetivos
	\begin{itemize}[itemsep=0em, topsep=0em]
		\vspace{-1em}
		\item Compreender os elementos constituintes da música e as propriedades do som;
		\item Conhecer os aspectos histórico-sociais da música nos períodos históricos, discutindo, sobretudo, as transformações na produção musical a partir do advento dos recursos fonográficos;
		\item Conhecer as principais ferramentas computacionais para a produção e criação musical;
		\item Desenvolver habilidades de produção e criação utilizando ferramentas computacionais.
	\end{itemize}
	
	\programa
	\begin{description}[itemsep=0em]
		\item[UNIDADE I:] Produção musical e tecnologia; 
	         \begin{enumerate}[itemsep=0em, topsep=0em]
				\item Parâmetros sons e elementos da Música;
				\begin{enumerate}
                	\item O som enquanto matéria da música;
                	\item Aspectos físicos do som: altura, intensidade, duração e timbre;
					\item Aspectos melódicos, rítmicos e harmônicos da música.
                \end{enumerate}
                
				\item Aspectos históricos e a construção da tradição da música ocidental.
				\begin{enumerate}
                	\item Períodos históricos da música e suas estéticas;
                	\item Música no século XX e XXI;
					\item Tradições e vanguardas na música europeia.
                \end{enumerate}
	        \end{enumerate}
	        
	    \item[UNIDADE II:] Aspectos criativos e tecnológicos em música;
	    	
	        \begin{enumerate}[itemsep=0em, topsep=0em]
				\setcounter{enumi}{2}
				\item Música e tecnologia.
				
				\begin{enumerate}[itemsep=0em, topsep=0em]
					\item Ferramentas computacionais para criação musical:
					
					\begin{enumerate}[itemsep=0em, topsep=0.25em]
                		\item Digital Audio Workstation (DAW);
    					\item Microfones;
    					\item Gravação de áudio;
    					\item Edição e mixagem.
                	\end{enumerate}
                	
				\end{enumerate}
				
	        \end{enumerate}
	        
	
	\end{description}
	
	\clearpage
	\metodologia
	As atividades serão desenvolvidas por meio de estudos teóricos e práticos. As aulas serão organizadas com base nas seguintes metodologias de ensino: (i) aulas expositivas; (ii) metodologias ativas de aprendizagem, como: debates, estudos dirigidos, jogos, criação de mapas mentais, entre outros; (iii) atividades de orientação de pesquisa, produção textual e apresentação oral; (iv) dinâmicas de criação e produção artística; (v) desenvolvimento de projetos integradores e interdisciplinares.

	
	\avaliacao
	A avaliação da aprendizagem na disciplina Artes será, parcialmente, realizada no decurso das aulas observando individualmente o gradual desenvolvimento dos alunos. A avaliação dar-se-á considerando a participação e produção dos alunos nas atividades propostas individualmente e em grupos durante todo o período letivo. O exercício da pesquisa será incentivado como ferramenta de construção do conhecimento. Assim, a produção da pesquisa, a produção textual e a apresentação oral em forma de seminário serão ferramentas de avaliação do trabalho desenvolvido. Serão considerados critérios avaliativos: (i) o envolvimento e a organização no processo de produção da pesquisa, (ii) a correção textual e o desenvolvimento argumentativo dos textos produzidos; (iii) desenvolvimento e organização da apresentação oral dos conteúdos pesquisados. 
	
		
	
	\begin{bibbasica}
			
		\item ALFONSO, Sandra Mara. \textbf{O violão, da marginalidade à academia}: trajetória de Jodacil Damaceno. Uberlândia, MG: EDUFU, 2009. ISBN 9788570781925.
		\item BENNETT, Roy. \textbf{Elementos Básicos Da Música}.Rio de Janeiro: Zahar, 2014. 98 p., il. ISBN 9788571101449. 
		\item BENNETT, Roy. \textbf{Uma Breve História Da Música}. Rio de Janeiro: Zahar, 2007. 79 p. (Cadernos de música da Universidade de Cambridge). ISBN 9788571103658. 
		\item VICENTE, Eduardo. \textbf{Da vitrola ao iPod}: uma história da indústria fonográfica no Brasil. Alameda Casa Editorial, 2014. ISBN 8579392055.
	
	\end{bibbasica}
	
	\begin{bibcomplementar}
    
		\item GRIFFITHS, Paul. \textbf{A música moderna}: uma história concisa e ilustrada de Debussy a Boulez. Zahar, 1987. ISBN 8571100047.
		\item SEVERIANO, Jairo. \textbf{Uma História da música popular brasileira}:  das origens à modernidade. 4 ed. São Paulo: Editora 34, 2017. 499 p. ISBN 97788573263961. 
		\item TABORDA, Marcia. \textbf{Violão e identidade nacional}: Rio de Janeiro, 1830-1930. Rio de Janeiro: Civilização Brasileira, 2011. 301 p. ISBN 9788520010297. 
		\item WITT, Stephen. \textbf{Como a música ficou grátis}: o fim de uma indústria, a virada do século e o paciente zero da pirataria. Rio de Janeiro: Intrínseca, 2015. ISBN 8580577705.
		\item ZUBEN, Paulo. \textbf{Música e tecnologia}: o som e seus novos instrumentos. Irmãos Vitale, 2004. ISBN 9788574071787. 
					
	\end{bibcomplementar}
	
	

\end{pud}


%%%% Educacao fisica
\begin{pud}

	\pudinfo{TSII.402}{Educação Física}{--}{15}{25}{2}{Optativa}
	
	\ementa
	A educação física no ensino técnico subsequente, que se caracteriza como o ciclo de aprofundamento e sistematização do conhecimento, tem como proposta despertar no aluno e aluna a compreensão de sujeito crítico capaz de intervir e modificar a realidade na qual se insere, bem como a valorização do seu corpo e da atividade física, por meio da ginástica, da dança, da luta, dos jogos e brincadeiras, do esporte, etc. Introdução ao processo de aquisição do conhecimento sistematizado acerca da cultura corporal. Desenvolvimento de reflexões, pesquisas e vivências da relação corpo, natureza e cultura e suas relações com a tecnologia. Princípios didático-pedagógicos para apropriação do conhecimento produzido e redimensionado pela humanidade ao longo de sua história.
	
	\objetivos
	\textbf{OBJETIVO GERAL}:
	Construir o conhecimento crítico-reflexivo sobre as práticas corporais, assegurando a participação irrestrita nas diversas vivências pertinentes à cultura corporal e sua relação com a área da informática.	
	\newline\\	
	\textbf{OBJETIVOS ESPECÍFICOS}:
	\begin{itemize}
		
		\item Conhecer as diversas manifestações da cultura corporal produzidas pelas diversas sociedades;
		\item Ressignificar as diversas manifestações da cultura corporal produzidas pelas diversas sociedades;
		\item  Vivenciar, de maneira teórica e prática, os elementos dos jogos, das danças, das lutas, das ginásticas, dos esportes e da qualidade de vida, atribuindo-lhes um sentido e um significado próprios;
		\item  Relacionar os conteúdos da educação física com a temática da tecnologia e sua atuação profissional no campo da informática;
		\item  Desenvolver atitudes e valores intrínsecos da cultura corporal, tais como ética, cooperação, liderança, autonomia, a criatividade, a integração, a capacidade de comunicação, reflexão, crítica, co-decisão e co-educação. 

	\end{itemize}
	
	\programa
	\begin{description}[itemsep=0em]
		\item[UNIDADE I:]\qquad 
	         \begin{enumerate}[itemsep=0em, topsep=0em]
				\item Manifestações da Cultura Corporal:
				\begin{itemize}
					\item Conhecimentos introdutórios sobre o corpo, saúde e qualidade de vida.
				\end{itemize}
	        \end{enumerate}
	        
	    \item[UNIDADE II:] \qquad
	         \begin{enumerate}[itemsep=0em, topsep=0em]
				\item Jogos, brinquedos e brincadeiras digitais;
				\item Lutas e jogos de oposição.
	        \end{enumerate}
	
		\item[UNIDADE III:] \qquad
	         \begin{enumerate}[itemsep=0em, topsep=0em]
				\item   Danças e atividades rítmicas;
				\item Ginástica e atividade física (Exergames).
	        \end{enumerate}
	                
	    \item[UNIDADE IV:] \qquad
	         \begin{enumerate}[itemsep=0em, topsep=0em]
				\item Esportes convencionais, não-convencionais e práticas corporais de aventura;
				\item Lazer, tempo livre e recreação.
	        \end{enumerate}
                                        

	\end{description}
	
	\metodologia
	\begin{enumerate}[itemsep=0em, topsep=0em]
		\vspace{-1em}
		\item  Aulas expositivas e dialogadas;
		\item Vivências práticas; 
		\item Produções textuais individuais e coletivas; 
		\item Leitura, interpretação e discussão de textos; 
		\item Exposições orais compartilhadas.
	\end{enumerate}
	
	\avaliacao
	\begin{description}[itemsep=0em, topsep=0em]
		\vspace{-1em}
		\item CONCEITUAL: Compreensão e apropriação dos conceitos, teorias e informações. 
		\begin{itemize}
			\item  Produções textuais;
			\item  Resolução de situações-problema;
			\item   Sínteses orais;
			\item   Pesquisa, síntese e apresentação.
		\end{itemize}
		
		\item PROCEDIMENTAL: Vivência, participação e desempenho crítico das atividades propostas. 
		\begin{itemize}
			\item  Participação efetiva; 
			\item   Envolvimento nos diversos momentos da aula; 
			\item   Criatividade e capacidade de ser co-autor do processo.
		\end{itemize}
		
		\item ATITUDINAL: Postura e atitude a nível pessoal e profissional: 
		\begin{itemize}
			\item  Atitudes que demonstrem companheirismo, ética, liderança e respeito (a si mesmo, aos demais e às regras).
		\end{itemize}
		
	\end{description}
	
	
	\begin{bibbasica}
			
		\item DARIDO, Suraya Cristina. \textbf{Para ensinar educação física}: possibilidades de intervenção na Escola. Campinas: Papirus, 2015. ISBN 9788530811556. (BVU)
		\item FINCK, Silvia Christina Madrid. \textbf{A Educação Física e o Esporte na Escola}: cotidiano, saberes e formação.  Curitiba: Editora Intersaberes, 2012. Livro. 188 p. ISBN 9788582120330. (BVU)
		\item MAFFEI, Willer Soares. \textbf{Introdução à formação em educação física}. Editora Intersaberes, 2017. 266 p. ISBN 9788559726015. (BVU) 
	
	\end{bibbasica}
	
	\begin{bibcomplementar}
    	\item CANO, Márcio Rogério de Oliveira; NEIRA, Marcos Garcia. \textbf{Educação física cultural}. Editora Blucher. Livro. 185  p. ISBN 9788521210443. (BVU)
		\item CASTELLANI FILHO, Lino. \textbf{Educação no Brasil}: a história que não se conta. 18. ed. Campinas: Papirus, 2010. ISBN 8530800214. (BVU)
		
		\item SANTOS, Ednei Fernando dos. \textbf{Manual de Primeiros Socorros da Educação Física aos Esportes}: o papel do educador físico no atendimento de socorro. Editora Interciência. Livro. 130 p. ISBN 9788563960085. (BVU)
		\item SILVA JÚNIOR, Vagner Pereira da. \textbf{Lazer e esporte no século XXI? Novidades no horizonte?}. Editora Intersaberes. Livro. 318 p. ISBN 9788559726930. (BVU)
		
		\item SILVA, Marcos Ruiz da; ALMEIDA, Bárbara Schausteck de; MICALISKI, Emerson Liomar. \textbf{Esportes complementares}. Editora Intersaberes. Livro. 226 p. ISBN 9788559729825. (BVU) % Disponível em: https://middleware-bv.am4.com.br/SSO/ifce/9788559729825. Acesso em: 9 Oct. 2020. 
					
	\end{bibcomplementar}
	
	
	
	
\end{pud}

%%% LIBRAS

\begin{pud}
	
	\pudinfo{TSII.403}{Libras}{--}{20}{20}{2}{Optativa}
	
	\ementa
	Concepção de linguagens de sinais. Linguagem de sinais brasileira. O código de ética. Resolução do encontro de Montevidéu. A formação de intérprete no mundo e no Brasil. Língua e identidade: um contexto de política linguística. Cultura surda e cidadania brasileira.
	
	\objetivos
	Compreender os principais aspectos da Língua Brasileira de Sinais -- Libras, língua oficial da comunidade surda brasileira, considerando a cultura surda, as identidades surdas, a história da surdez, a legislação vigente e o uso da língua.
	
	\programa
	\begin{description}[itemsep=0em]
		\item[UNIDADE I:] Introdução a Libras; 
	         \begin{enumerate}[itemsep=0em, topsep=0em]
				\item  Os surdos na Antiguidade;
				\item  O surdo na Idade Moderna;
				\item  O surdo na idade contemporânea;
				\item  O surdo do século XX;
				\item  Fundamentação Legal da Libras;
				\item  Conceito de Linguagem;
				\item  Parâmetros da LIBRAS;
				\item  Diálogos em LIBRAS;
				\item  Alfabeto Manual e Numeral;
				\item  Calendário em LIBRAS;
				\item  Pessoas/Família;
				\item  Documentos;
				\item  Pronomes; 
				\item Lugares;
				\item Natureza; 
				\item Cores; 
				\item Escola; 
				\item Casa; 
				\item Alimento. 
	        \end{enumerate}
	        
	    \item[UNIDADE II:] Libras no dia a dia;
	         \begin{enumerate}[itemsep=0em, topsep=0em]
				\item   Bebidas;
				\item  Vestuários/Objetos Pessoais;
				\item  Profissões;
				\item Animais;
				\item  Corpo Humano;
				\item  Higiene;
				\item  Saúde;
				\item  Meios de Transporte;
				\item  Meios de comunicação;
				\item  Lazer/Esporte;
				\item  Instrumentos Musicais.
	        \end{enumerate}
	
		\item[UNIDADE III:] Português da Libras.
	         \begin{enumerate}[itemsep=0em, topsep=0em]
				\item Verbos;
				\item Negativos;
				\item Adjetivos/Advérbios;
				\item Atividades Escritas e Oral;
				\item O código de ética do interprete;
				\item A formação de interprete no mundo e no Brasil.
	        \end{enumerate}
	                                
	
	\end{description}
	
	\metodologia
	Serão aplicadas técnicas de exposição dialogada, dinâmica de grupo, pesquisa bibliográfica, apresentação e discussão de filmes; produção de texto, seminários, trabalhos individuais e em grupo.
	
	\avaliacao
	A  avaliação é realizada de forma processual e cumulativa utilizando os instrumentos de avaliação especificados pelo Regulamento de Organização Didática em seu art. 94 § 1$^\circ$, conforme for mais adequado. A frequência é obrigatória, respeitando os limites de ausência previstos em lei.
	
	
	\begin{bibbasica}
		\item FERNANDES, Sueli. \textbf{Educação de surdos}. Curitiba: InterSaberes, 2012. Livro.  144 p. ISBN 9788582120149. (BVU)
		%\item GESSER, Andrei. \textbf{Libras? Que língua é essa?}: crenças e preconceitos em torno da língua de sinais e da realidade surda. São Paulo: Parábola, 2009. ISBN 9788579340017.
		\item QUADROS, Ronice Muller. \textbf{Língua de sinais brasileira}: estudos linguísticos. Volume único. Porto Alegre: Artmed. 2004. ISBN  8536303085.		
		\item SACKS, Oliver W. \textbf{Vendo vozes}: uma viagem ao mundo dos surdos.   São Paulo: Companhia das Letras, 2015. 215 p. ISBN 978-85-359-1608-9.
	
	\end{bibbasica}
	
	\clearpage
	\begin{bibcomplementar}
    
		\item BAGGIO, Maria Auxiliadora; CASA NOVA, Maria da Graça. \textbf{Libras}. Editora Intersaberes, 2017. Livro. 146 p. ISBN 9788544301890. (BVU)
		%\item BRASIL. \textbf{Decreto 5.626 de 22 de dezembro de 2005}. Brasília. 2005. Disponível em: \url{http://www.planalto.gov.br/ccivil_03/_ato2004-2006/2005/decreto/d5626.htm}. Acesso em: 16 ago. 2018.
		\item PEREIRA, Maria Cristina da Cunha (Org.). \textbf{Libras}: conhecimento além dos sinais.  Editora Pearson. Livro.  146 p. ISBN 9788576058786. (BVU)
		\item QUADROS, Ronice Muller de. \textbf{Educação de surdos}: a aquisição da linguagem. Porto Alegre: Artmed, 2008. 126 p. ISBN 9788573072655.
		\item SANTANA, Ana Paula. \textbf{Surdez e linguagem}: aspectos e aplicações. 5. ed. São Paulo: Summus Editorial, 2015.  Livro. 328 p. ISBN 9788585689971. (BVU) 
		\item SILVA, Rafael Dias (Org.). \textbf{Língua brasileira de sinais}: libras.  Editora Pearson. Livro. 218 p. ISBN 9788543016733. (BVU)
		
		%\item HONORA, Márcia. \textbf{Livro ilustrado de Língua Brasileira de Sinais}: desvendando a comunicação usada pelas pessoas com surdez. Colaboração de Mary Lopes Esteves Frizanco. São Paulo: Ciranda Cultural, 2009. ISBN 9788538014218
		%\item SKLIAR, Carlos Obra: \textbf{A Surdez}: um olhar sobre as diferenças. Porto Alegre: Mediação. 1998.
	
	\end{bibcomplementar}		
	
		
\end{pud}